
% Default to the notebook output style

    


% Inherit from the specified cell style.




    
\documentclass[11pt]{article}

    
    
    \usepackage[T1]{fontenc}
    % Nicer default font (+ math font) than Computer Modern for most use cases
    \usepackage{mathpazo}

    % Basic figure setup, for now with no caption control since it's done
    % automatically by Pandoc (which extracts ![](path) syntax from Markdown).
    \usepackage{graphicx}
    % We will generate all images so they have a width \maxwidth. This means
    % that they will get their normal width if they fit onto the page, but
    % are scaled down if they would overflow the margins.
    \makeatletter
    \def\maxwidth{\ifdim\Gin@nat@width>\linewidth\linewidth
    \else\Gin@nat@width\fi}
    \makeatother
    \let\Oldincludegraphics\includegraphics
    % Set max figure width to be 80% of text width, for now hardcoded.
    \renewcommand{\includegraphics}[1]{\Oldincludegraphics[width=.8\maxwidth]{#1}}
    % Ensure that by default, figures have no caption (until we provide a
    % proper Figure object with a Caption API and a way to capture that
    % in the conversion process - todo).
    \usepackage{caption}
    \DeclareCaptionLabelFormat{nolabel}{}
    \captionsetup{labelformat=nolabel}

    \usepackage{adjustbox} % Used to constrain images to a maximum size 
    \usepackage{xcolor} % Allow colors to be defined
    \usepackage{enumerate} % Needed for markdown enumerations to work
    \usepackage{geometry} % Used to adjust the document margins
    \usepackage{amsmath} % Equations
    \usepackage{amssymb} % Equations
    \usepackage{textcomp} % defines textquotesingle
    % Hack from http://tex.stackexchange.com/a/47451/13684:
    \AtBeginDocument{%
        \def\PYZsq{\textquotesingle}% Upright quotes in Pygmentized code
    }
    \usepackage{upquote} % Upright quotes for verbatim code
    \usepackage{eurosym} % defines \euro
    \usepackage[mathletters]{ucs} % Extended unicode (utf-8) support
    \usepackage[utf8x]{inputenc} % Allow utf-8 characters in the tex document
    \usepackage{fancyvrb} % verbatim replacement that allows latex
    \usepackage{grffile} % extends the file name processing of package graphics 
                         % to support a larger range 
    % The hyperref package gives us a pdf with properly built
    % internal navigation ('pdf bookmarks' for the table of contents,
    % internal cross-reference links, web links for URLs, etc.)
    \usepackage{hyperref}
    \usepackage{longtable} % longtable support required by pandoc >1.10
    \usepackage{booktabs}  % table support for pandoc > 1.12.2
    \usepackage[inline]{enumitem} % IRkernel/repr support (it uses the enumerate* environment)
    \usepackage[normalem]{ulem} % ulem is needed to support strikethroughs (\sout)
                                % normalem makes italics be italics, not underlines
    

    
    
    % Colors for the hyperref package
    \definecolor{urlcolor}{rgb}{0,.145,.698}
    \definecolor{linkcolor}{rgb}{.71,0.21,0.01}
    \definecolor{citecolor}{rgb}{.12,.54,.11}

    % ANSI colors
    \definecolor{ansi-black}{HTML}{3E424D}
    \definecolor{ansi-black-intense}{HTML}{282C36}
    \definecolor{ansi-red}{HTML}{E75C58}
    \definecolor{ansi-red-intense}{HTML}{B22B31}
    \definecolor{ansi-green}{HTML}{00A250}
    \definecolor{ansi-green-intense}{HTML}{007427}
    \definecolor{ansi-yellow}{HTML}{DDB62B}
    \definecolor{ansi-yellow-intense}{HTML}{B27D12}
    \definecolor{ansi-blue}{HTML}{208FFB}
    \definecolor{ansi-blue-intense}{HTML}{0065CA}
    \definecolor{ansi-magenta}{HTML}{D160C4}
    \definecolor{ansi-magenta-intense}{HTML}{A03196}
    \definecolor{ansi-cyan}{HTML}{60C6C8}
    \definecolor{ansi-cyan-intense}{HTML}{258F8F}
    \definecolor{ansi-white}{HTML}{C5C1B4}
    \definecolor{ansi-white-intense}{HTML}{A1A6B2}

    % commands and environments needed by pandoc snippets
    % extracted from the output of `pandoc -s`
    \providecommand{\tightlist}{%
      \setlength{\itemsep}{0pt}\setlength{\parskip}{0pt}}
    \DefineVerbatimEnvironment{Highlighting}{Verbatim}{commandchars=\\\{\}}
    % Add ',fontsize=\small' for more characters per line
    \newenvironment{Shaded}{}{}
    \newcommand{\KeywordTok}[1]{\textcolor[rgb]{0.00,0.44,0.13}{\textbf{{#1}}}}
    \newcommand{\DataTypeTok}[1]{\textcolor[rgb]{0.56,0.13,0.00}{{#1}}}
    \newcommand{\DecValTok}[1]{\textcolor[rgb]{0.25,0.63,0.44}{{#1}}}
    \newcommand{\BaseNTok}[1]{\textcolor[rgb]{0.25,0.63,0.44}{{#1}}}
    \newcommand{\FloatTok}[1]{\textcolor[rgb]{0.25,0.63,0.44}{{#1}}}
    \newcommand{\CharTok}[1]{\textcolor[rgb]{0.25,0.44,0.63}{{#1}}}
    \newcommand{\StringTok}[1]{\textcolor[rgb]{0.25,0.44,0.63}{{#1}}}
    \newcommand{\CommentTok}[1]{\textcolor[rgb]{0.38,0.63,0.69}{\textit{{#1}}}}
    \newcommand{\OtherTok}[1]{\textcolor[rgb]{0.00,0.44,0.13}{{#1}}}
    \newcommand{\AlertTok}[1]{\textcolor[rgb]{1.00,0.00,0.00}{\textbf{{#1}}}}
    \newcommand{\FunctionTok}[1]{\textcolor[rgb]{0.02,0.16,0.49}{{#1}}}
    \newcommand{\RegionMarkerTok}[1]{{#1}}
    \newcommand{\ErrorTok}[1]{\textcolor[rgb]{1.00,0.00,0.00}{\textbf{{#1}}}}
    \newcommand{\NormalTok}[1]{{#1}}
    
    % Additional commands for more recent versions of Pandoc
    \newcommand{\ConstantTok}[1]{\textcolor[rgb]{0.53,0.00,0.00}{{#1}}}
    \newcommand{\SpecialCharTok}[1]{\textcolor[rgb]{0.25,0.44,0.63}{{#1}}}
    \newcommand{\VerbatimStringTok}[1]{\textcolor[rgb]{0.25,0.44,0.63}{{#1}}}
    \newcommand{\SpecialStringTok}[1]{\textcolor[rgb]{0.73,0.40,0.53}{{#1}}}
    \newcommand{\ImportTok}[1]{{#1}}
    \newcommand{\DocumentationTok}[1]{\textcolor[rgb]{0.73,0.13,0.13}{\textit{{#1}}}}
    \newcommand{\AnnotationTok}[1]{\textcolor[rgb]{0.38,0.63,0.69}{\textbf{\textit{{#1}}}}}
    \newcommand{\CommentVarTok}[1]{\textcolor[rgb]{0.38,0.63,0.69}{\textbf{\textit{{#1}}}}}
    \newcommand{\VariableTok}[1]{\textcolor[rgb]{0.10,0.09,0.49}{{#1}}}
    \newcommand{\ControlFlowTok}[1]{\textcolor[rgb]{0.00,0.44,0.13}{\textbf{{#1}}}}
    \newcommand{\OperatorTok}[1]{\textcolor[rgb]{0.40,0.40,0.40}{{#1}}}
    \newcommand{\BuiltInTok}[1]{{#1}}
    \newcommand{\ExtensionTok}[1]{{#1}}
    \newcommand{\PreprocessorTok}[1]{\textcolor[rgb]{0.74,0.48,0.00}{{#1}}}
    \newcommand{\AttributeTok}[1]{\textcolor[rgb]{0.49,0.56,0.16}{{#1}}}
    \newcommand{\InformationTok}[1]{\textcolor[rgb]{0.38,0.63,0.69}{\textbf{\textit{{#1}}}}}
    \newcommand{\WarningTok}[1]{\textcolor[rgb]{0.38,0.63,0.69}{\textbf{\textit{{#1}}}}}
    
    
    % Define a nice break command that doesn't care if a line doesn't already
    % exist.
    \def\br{\hspace*{\fill} \\* }
    % Math Jax compatability definitions
    \def\gt{>}
    \def\lt{<}
    % Document parameters
    \title{demo6\_workbook}
    
    
    

    % Pygments definitions
    
\makeatletter
\def\PY@reset{\let\PY@it=\relax \let\PY@bf=\relax%
    \let\PY@ul=\relax \let\PY@tc=\relax%
    \let\PY@bc=\relax \let\PY@ff=\relax}
\def\PY@tok#1{\csname PY@tok@#1\endcsname}
\def\PY@toks#1+{\ifx\relax#1\empty\else%
    \PY@tok{#1}\expandafter\PY@toks\fi}
\def\PY@do#1{\PY@bc{\PY@tc{\PY@ul{%
    \PY@it{\PY@bf{\PY@ff{#1}}}}}}}
\def\PY#1#2{\PY@reset\PY@toks#1+\relax+\PY@do{#2}}

\expandafter\def\csname PY@tok@w\endcsname{\def\PY@tc##1{\textcolor[rgb]{0.73,0.73,0.73}{##1}}}
\expandafter\def\csname PY@tok@c\endcsname{\let\PY@it=\textit\def\PY@tc##1{\textcolor[rgb]{0.25,0.50,0.50}{##1}}}
\expandafter\def\csname PY@tok@cp\endcsname{\def\PY@tc##1{\textcolor[rgb]{0.74,0.48,0.00}{##1}}}
\expandafter\def\csname PY@tok@k\endcsname{\let\PY@bf=\textbf\def\PY@tc##1{\textcolor[rgb]{0.00,0.50,0.00}{##1}}}
\expandafter\def\csname PY@tok@kp\endcsname{\def\PY@tc##1{\textcolor[rgb]{0.00,0.50,0.00}{##1}}}
\expandafter\def\csname PY@tok@kt\endcsname{\def\PY@tc##1{\textcolor[rgb]{0.69,0.00,0.25}{##1}}}
\expandafter\def\csname PY@tok@o\endcsname{\def\PY@tc##1{\textcolor[rgb]{0.40,0.40,0.40}{##1}}}
\expandafter\def\csname PY@tok@ow\endcsname{\let\PY@bf=\textbf\def\PY@tc##1{\textcolor[rgb]{0.67,0.13,1.00}{##1}}}
\expandafter\def\csname PY@tok@nb\endcsname{\def\PY@tc##1{\textcolor[rgb]{0.00,0.50,0.00}{##1}}}
\expandafter\def\csname PY@tok@nf\endcsname{\def\PY@tc##1{\textcolor[rgb]{0.00,0.00,1.00}{##1}}}
\expandafter\def\csname PY@tok@nc\endcsname{\let\PY@bf=\textbf\def\PY@tc##1{\textcolor[rgb]{0.00,0.00,1.00}{##1}}}
\expandafter\def\csname PY@tok@nn\endcsname{\let\PY@bf=\textbf\def\PY@tc##1{\textcolor[rgb]{0.00,0.00,1.00}{##1}}}
\expandafter\def\csname PY@tok@ne\endcsname{\let\PY@bf=\textbf\def\PY@tc##1{\textcolor[rgb]{0.82,0.25,0.23}{##1}}}
\expandafter\def\csname PY@tok@nv\endcsname{\def\PY@tc##1{\textcolor[rgb]{0.10,0.09,0.49}{##1}}}
\expandafter\def\csname PY@tok@no\endcsname{\def\PY@tc##1{\textcolor[rgb]{0.53,0.00,0.00}{##1}}}
\expandafter\def\csname PY@tok@nl\endcsname{\def\PY@tc##1{\textcolor[rgb]{0.63,0.63,0.00}{##1}}}
\expandafter\def\csname PY@tok@ni\endcsname{\let\PY@bf=\textbf\def\PY@tc##1{\textcolor[rgb]{0.60,0.60,0.60}{##1}}}
\expandafter\def\csname PY@tok@na\endcsname{\def\PY@tc##1{\textcolor[rgb]{0.49,0.56,0.16}{##1}}}
\expandafter\def\csname PY@tok@nt\endcsname{\let\PY@bf=\textbf\def\PY@tc##1{\textcolor[rgb]{0.00,0.50,0.00}{##1}}}
\expandafter\def\csname PY@tok@nd\endcsname{\def\PY@tc##1{\textcolor[rgb]{0.67,0.13,1.00}{##1}}}
\expandafter\def\csname PY@tok@s\endcsname{\def\PY@tc##1{\textcolor[rgb]{0.73,0.13,0.13}{##1}}}
\expandafter\def\csname PY@tok@sd\endcsname{\let\PY@it=\textit\def\PY@tc##1{\textcolor[rgb]{0.73,0.13,0.13}{##1}}}
\expandafter\def\csname PY@tok@si\endcsname{\let\PY@bf=\textbf\def\PY@tc##1{\textcolor[rgb]{0.73,0.40,0.53}{##1}}}
\expandafter\def\csname PY@tok@se\endcsname{\let\PY@bf=\textbf\def\PY@tc##1{\textcolor[rgb]{0.73,0.40,0.13}{##1}}}
\expandafter\def\csname PY@tok@sr\endcsname{\def\PY@tc##1{\textcolor[rgb]{0.73,0.40,0.53}{##1}}}
\expandafter\def\csname PY@tok@ss\endcsname{\def\PY@tc##1{\textcolor[rgb]{0.10,0.09,0.49}{##1}}}
\expandafter\def\csname PY@tok@sx\endcsname{\def\PY@tc##1{\textcolor[rgb]{0.00,0.50,0.00}{##1}}}
\expandafter\def\csname PY@tok@m\endcsname{\def\PY@tc##1{\textcolor[rgb]{0.40,0.40,0.40}{##1}}}
\expandafter\def\csname PY@tok@gh\endcsname{\let\PY@bf=\textbf\def\PY@tc##1{\textcolor[rgb]{0.00,0.00,0.50}{##1}}}
\expandafter\def\csname PY@tok@gu\endcsname{\let\PY@bf=\textbf\def\PY@tc##1{\textcolor[rgb]{0.50,0.00,0.50}{##1}}}
\expandafter\def\csname PY@tok@gd\endcsname{\def\PY@tc##1{\textcolor[rgb]{0.63,0.00,0.00}{##1}}}
\expandafter\def\csname PY@tok@gi\endcsname{\def\PY@tc##1{\textcolor[rgb]{0.00,0.63,0.00}{##1}}}
\expandafter\def\csname PY@tok@gr\endcsname{\def\PY@tc##1{\textcolor[rgb]{1.00,0.00,0.00}{##1}}}
\expandafter\def\csname PY@tok@ge\endcsname{\let\PY@it=\textit}
\expandafter\def\csname PY@tok@gs\endcsname{\let\PY@bf=\textbf}
\expandafter\def\csname PY@tok@gp\endcsname{\let\PY@bf=\textbf\def\PY@tc##1{\textcolor[rgb]{0.00,0.00,0.50}{##1}}}
\expandafter\def\csname PY@tok@go\endcsname{\def\PY@tc##1{\textcolor[rgb]{0.53,0.53,0.53}{##1}}}
\expandafter\def\csname PY@tok@gt\endcsname{\def\PY@tc##1{\textcolor[rgb]{0.00,0.27,0.87}{##1}}}
\expandafter\def\csname PY@tok@err\endcsname{\def\PY@bc##1{\setlength{\fboxsep}{0pt}\fcolorbox[rgb]{1.00,0.00,0.00}{1,1,1}{\strut ##1}}}
\expandafter\def\csname PY@tok@kc\endcsname{\let\PY@bf=\textbf\def\PY@tc##1{\textcolor[rgb]{0.00,0.50,0.00}{##1}}}
\expandafter\def\csname PY@tok@kd\endcsname{\let\PY@bf=\textbf\def\PY@tc##1{\textcolor[rgb]{0.00,0.50,0.00}{##1}}}
\expandafter\def\csname PY@tok@kn\endcsname{\let\PY@bf=\textbf\def\PY@tc##1{\textcolor[rgb]{0.00,0.50,0.00}{##1}}}
\expandafter\def\csname PY@tok@kr\endcsname{\let\PY@bf=\textbf\def\PY@tc##1{\textcolor[rgb]{0.00,0.50,0.00}{##1}}}
\expandafter\def\csname PY@tok@bp\endcsname{\def\PY@tc##1{\textcolor[rgb]{0.00,0.50,0.00}{##1}}}
\expandafter\def\csname PY@tok@fm\endcsname{\def\PY@tc##1{\textcolor[rgb]{0.00,0.00,1.00}{##1}}}
\expandafter\def\csname PY@tok@vc\endcsname{\def\PY@tc##1{\textcolor[rgb]{0.10,0.09,0.49}{##1}}}
\expandafter\def\csname PY@tok@vg\endcsname{\def\PY@tc##1{\textcolor[rgb]{0.10,0.09,0.49}{##1}}}
\expandafter\def\csname PY@tok@vi\endcsname{\def\PY@tc##1{\textcolor[rgb]{0.10,0.09,0.49}{##1}}}
\expandafter\def\csname PY@tok@vm\endcsname{\def\PY@tc##1{\textcolor[rgb]{0.10,0.09,0.49}{##1}}}
\expandafter\def\csname PY@tok@sa\endcsname{\def\PY@tc##1{\textcolor[rgb]{0.73,0.13,0.13}{##1}}}
\expandafter\def\csname PY@tok@sb\endcsname{\def\PY@tc##1{\textcolor[rgb]{0.73,0.13,0.13}{##1}}}
\expandafter\def\csname PY@tok@sc\endcsname{\def\PY@tc##1{\textcolor[rgb]{0.73,0.13,0.13}{##1}}}
\expandafter\def\csname PY@tok@dl\endcsname{\def\PY@tc##1{\textcolor[rgb]{0.73,0.13,0.13}{##1}}}
\expandafter\def\csname PY@tok@s2\endcsname{\def\PY@tc##1{\textcolor[rgb]{0.73,0.13,0.13}{##1}}}
\expandafter\def\csname PY@tok@sh\endcsname{\def\PY@tc##1{\textcolor[rgb]{0.73,0.13,0.13}{##1}}}
\expandafter\def\csname PY@tok@s1\endcsname{\def\PY@tc##1{\textcolor[rgb]{0.73,0.13,0.13}{##1}}}
\expandafter\def\csname PY@tok@mb\endcsname{\def\PY@tc##1{\textcolor[rgb]{0.40,0.40,0.40}{##1}}}
\expandafter\def\csname PY@tok@mf\endcsname{\def\PY@tc##1{\textcolor[rgb]{0.40,0.40,0.40}{##1}}}
\expandafter\def\csname PY@tok@mh\endcsname{\def\PY@tc##1{\textcolor[rgb]{0.40,0.40,0.40}{##1}}}
\expandafter\def\csname PY@tok@mi\endcsname{\def\PY@tc##1{\textcolor[rgb]{0.40,0.40,0.40}{##1}}}
\expandafter\def\csname PY@tok@il\endcsname{\def\PY@tc##1{\textcolor[rgb]{0.40,0.40,0.40}{##1}}}
\expandafter\def\csname PY@tok@mo\endcsname{\def\PY@tc##1{\textcolor[rgb]{0.40,0.40,0.40}{##1}}}
\expandafter\def\csname PY@tok@ch\endcsname{\let\PY@it=\textit\def\PY@tc##1{\textcolor[rgb]{0.25,0.50,0.50}{##1}}}
\expandafter\def\csname PY@tok@cm\endcsname{\let\PY@it=\textit\def\PY@tc##1{\textcolor[rgb]{0.25,0.50,0.50}{##1}}}
\expandafter\def\csname PY@tok@cpf\endcsname{\let\PY@it=\textit\def\PY@tc##1{\textcolor[rgb]{0.25,0.50,0.50}{##1}}}
\expandafter\def\csname PY@tok@c1\endcsname{\let\PY@it=\textit\def\PY@tc##1{\textcolor[rgb]{0.25,0.50,0.50}{##1}}}
\expandafter\def\csname PY@tok@cs\endcsname{\let\PY@it=\textit\def\PY@tc##1{\textcolor[rgb]{0.25,0.50,0.50}{##1}}}

\def\PYZbs{\char`\\}
\def\PYZus{\char`\_}
\def\PYZob{\char`\{}
\def\PYZcb{\char`\}}
\def\PYZca{\char`\^}
\def\PYZam{\char`\&}
\def\PYZlt{\char`\<}
\def\PYZgt{\char`\>}
\def\PYZsh{\char`\#}
\def\PYZpc{\char`\%}
\def\PYZdl{\char`\$}
\def\PYZhy{\char`\-}
\def\PYZsq{\char`\'}
\def\PYZdq{\char`\"}
\def\PYZti{\char`\~}
% for compatibility with earlier versions
\def\PYZat{@}
\def\PYZlb{[}
\def\PYZrb{]}
\makeatother


    % Exact colors from NB
    \definecolor{incolor}{rgb}{0.0, 0.0, 0.5}
    \definecolor{outcolor}{rgb}{0.545, 0.0, 0.0}



    
    % Prevent overflowing lines due to hard-to-break entities
    \sloppy 
    % Setup hyperref package
    \hypersetup{
      breaklinks=true,  % so long urls are correctly broken across lines
      colorlinks=true,
      urlcolor=urlcolor,
      linkcolor=linkcolor,
      citecolor=citecolor,
      }
    % Slightly bigger margins than the latex defaults
    
    \geometry{verbose,tmargin=1in,bmargin=1in,lmargin=1in,rmargin=1in}
    
    

    \begin{document}
    
    
    \maketitle
    
    

    
    \hypertarget{wk6-demo---supervised-learning-gradient-descent}{%
\section{wk6 Demo - Supervised Learning \& Gradient
Descent}\label{wk6-demo---supervised-learning-gradient-descent}}

\textbf{\texttt{MIDS\ w261:\ Machine\ Learning\ at\ Scale\ \textbar{}\ UC\ Berkeley\ School\ of\ Information\ \textbar{}\ Spring\ 2019}}

In Supervised Machine Learning we use labeled training data to learn a
decision function (a.k.a `model') and make evaluations about how well
that decision function might perform when applied to new data. Of course
the biggest factor that will determine the performance of your model is
the quality of the data you train on. However another key challenge is
the question of what models to consider \& how to compare their
performance so that you can choose the best one. Gradient Descent solves
this challenge for a certain class of functions. By the end of this live
session you should be able to: * \textbf{\ldots{} define} the loss
function for OLS Regression and its gradient. * \textbf{\ldots{}
explain} the relationship between model space and parameter space. *
\textbf{\ldots{} recognize} convex optimization problems and why they
are desirable. * \textbf{\ldots{} describe} the process of Gradient
Descent \& how it can be parallelized.

    \hypertarget{introduction}{%
\section{Introduction}\label{introduction}}

In today's demo, we'll use Linear Regression on a simple example in
order to explore key topics related to distributed learning of
parametric models. Broadly speaking, the supervised learning of a
parametric model can be split into to two components:

\begin{enumerate}
\def\labelenumi{\arabic{enumi}.}
\tightlist
\item
  \textbf{Optimization Task (a.k.a. Learning)}: Given a strategy for
  making a prediction, return the specific parameters which guarantee
  the optimal prediction.\\
\item
  \textbf{Prediction Task}: Given an input vector, return an output
  value.
\end{enumerate}

\begin{quote}
\textbf{DISCUSSION QUESTION:} \emph{In the case of Linear Regression,
which of the two tasks above are we most likely to want to parallelize?
Why?}
\end{quote}

OK, Let's start with a quick review of some notation you will have seen
in w207.

\hypertarget{notation-review}{%
\subsection{Notation Review}\label{notation-review}}

Linear Regression tackles the \textbf{prediction task} by assuming that
we can compute our output variable, \(y\), using a linear combination of
our input variables. That is we assume there exist a set of
\textbf{weights}, \(\mathbf{w}\), and a \textbf{bias} term,
\(\mathbf{b}\), such that for any input
\(\mathbf{x}_j \in \mathbb{R}^m\):

\begin{equation}\tag{1.1}
y_j = \displaystyle\sum_{i=1}^{m}{w_i\cdot x_{ji} + b}
\end{equation}

In vector notation, this can be written:

\begin{equation}
y_j = \displaystyle{\mathbf{w}^T\mathbf{x}_{j} + b}
\end{equation}

Of course, this perfect linear relationship never holds over a whole
dataset \textbf{\(X\)}, so Linear Regression attempts to fit (i.e.
\textbf{learn}) the best line (in 1 dimension) or hyperplane (in 2 or
more dimensions) to the data. In the case of \textbf{ordinary least
squares (OLS)} linear regression, best fit is defined as minimizing the
Euclidean distances of each point in the dataset to the line or
hyperplane. These distances are often referred to as \textbf{residuals}.

    \begin{Verbatim}[commandchars=\\\{\}]
{\color{incolor}In [{\color{incolor}1}]:} \PY{k+kn}{from} \PY{n+nn}{IPython}\PY{n+nn}{.}\PY{n+nn}{display} \PY{k}{import} \PY{n}{Image}
        \PY{n}{Image}\PY{p}{(}\PY{n}{filename}\PY{o}{=}\PY{l+s+s2}{\PYZdq{}}\PY{l+s+s2}{residual.png}\PY{l+s+s2}{\PYZdq{}}\PY{p}{,} \PY{n}{width}\PY{o}{=}\PY{l+s+s2}{\PYZdq{}}\PY{l+s+s2}{400}\PY{l+s+s2}{\PYZdq{}}\PY{p}{,} \PY{n}{height}\PY{o}{=}\PY{l+s+s2}{\PYZdq{}}\PY{l+s+s2}{200}\PY{l+s+s2}{\PYZdq{}}\PY{p}{)}
\end{Verbatim}

\texttt{\color{outcolor}Out[{\color{outcolor}1}]:}
    
    \begin{center}
    \adjustimage{max size={0.9\linewidth}{0.9\paperheight}}{output_2_0.png}
    \end{center}
    { \hspace*{\fill} \\}
    

    The calculation of the average residual (a.k.a.\textbf{mean squared
error, MSE}) over our test or training set allows us to measure how good
a fit we've achieved. We call this function the \textbf{loss} or
\textbf{objective} function because our goal in the \textbf{optimization
task} is to find the parameters which minimize it. (Ok, yes,
\emph{technically} MSE is \emph{not actually equal} to the average
residual but it is conceptually equivalent \& guaranteed to have the
same minimum.)

\begin{equation}\tag{1.2}
f(\mathbf{w}, b) = \frac{1}{n}\sum_{j=1}^{n}\left[ (\mathbf{w}^T\mathbf{x}_j + b) - y_i\right]^2,\\
n = \left|X_{\text{train}}\right|
\end{equation}

For convenience, we sometimes choose to think of the bias \(b\) as
weight \(w_{m+1}\). To operationalize this, we'll \emph{augment} our
input vectors by setting \(x_{m+1}=1\). This gives us a simpler way to
write the loss function: \[
\mathbf{x}' :=
\begin{bmatrix}
\mathbf{x}\\
1
\end{bmatrix},\quad
\boldsymbol{\theta} :=
\begin{bmatrix}
\mathbf{w}\\
b
\end{bmatrix}
\]

\begin{equation}\tag{1.3}
f(\boldsymbol{\theta}) = \frac{1}{n}\sum_{i=1}^{n}\left[ \boldsymbol{\theta}^T\cdot\mathbf{x}'_i - y_i\right]^2
\end{equation}

Machine Learning packages like \texttt{sklearn} and \texttt{tensorflow}
take this one step further by representing the entire training set in a
single matrix were each row is an input vector and each column
represents a feature: \[
\text{X}' =
\begin{bmatrix}
\mathbf{x'}_1^{\text{T}}\\
\vdots\\
\mathbf{x'}_n^{\text{T}}
\end{bmatrix},\quad
\mathbf{y} = 
\begin{bmatrix}
y_1\\
\vdots\\
y_n
\end{bmatrix}
\]

\begin{equation}\tag{1.4}
f(\boldsymbol{\theta}) = \frac{1}{n}\|\text{X}'\cdot \boldsymbol{\theta} - \mathbf{y}\|_2^2
\end{equation}

As you see here, it is customary to write loss as a function of the
parameters \(\theta\) (or equivalently \(\mathbf{w}\) and \(b\)).
However it is important to note that the MSE loss depends on both the
parameters/weights \emph{and} the data \(X\), we'll talk more about that
later.

    \begin{quote}
\textbf{DISCUSSION QUESTIONS:} * \emph{In equation 1.1 what do
\(x_{ji}\), \(w_i\), and \(\mathbf{w}\) each represent?}\\
* \emph{In the asynch's version of the loss function \(\alpha\) and
\(\beta\) appear as parameters\ldots{} what do they represent? How are
they captured in equations 1.2 and 1.3 respectively?} * \emph{If we were
computing loss over a really large data set what might be the arguments
in favor / against using the augmented version of the loss function
calculation?}
\end{quote}

    \hypertarget{solution}{%
\subsubsection{\textless{}--- SOLUTION
---\textgreater{}}\label{solution}}

\textbf{INSTRUCTOR TALKING POINTS}\\
* \emph{In equation 1.1 what do \(x_{ji}\), \(w_i\), and \(\mathbf{w}\)
each represent?} \textgreater{} \(x_{ij}\) is the \(i^{th}\) variable in
the \(j^{th}\) data example. \(w_i\) is the \(i^{th}\) weight
(parameter), and \(\mathbf{w}\) is the entire weight (parameter) vector.
* \emph{In the asynch's version of the loss function \(\alpha\) and
\(\beta\) appear as parameters\ldots{} what do they represent? How are
they captured in equations 1.2 and 1.3 respectively?} \textgreater{}
\(\alpha\) represents the weights and \(\beta\) the bias, in equation
1.2 these are \(w\) and \(b\), in equation 1.3 we append \(b\) to the
weights vecotor to get an augmented weight vector \(\theta\) which is
just another way of representing \(\alpha\) and \(\beta\). NOTE -- the
greek letter \(\alpha\) is also often (and confusingly) used to
represent the learning rate in gradient descent - we'll try to use
\(\eta\) ``eta'' instead to avoid confusion, but its something to keep
an eye out for. * \emph{If we were computing loss over a really large
data set what might be the arguments in favor / against using the
augmented version of the loss function calculation?} \textgreater{}
Having to augment the entire data set prior to learning adds an
additional pass over the data. In addition, it doubles the storage
required. Instead, we can ``augment'' each example as we encounter it.

    \hypertarget{a-warning-about-ols-before-we-start}{%
\paragraph{A warning about OLS before we
start:}\label{a-warning-about-ols-before-we-start}}

Supervised learning models, especially interprable ones, and especially
linear/logistic regression, tend to get used for two different kinds of
tasks: prediction and inference -- it is important to remember the
difference between these two use cases. While it is practically possible
to fit a linear model to any dataset and then use that model to make
predictions\ldots{} it is \emph{not} always fair to use the coefficients
of your model to infer relationships (causal or otherwise) between your
features and outcome variable. As you will rememeber from w203 and w207
if you are going to perform inference using OLS, your data should
satisfy the following conditions: 1. Residuals are homoscedastic - they
have constant variance\\
1. Residuals are normaly distributed 1. No multicolinearity - features
are not correlated

\textbf{For more info see the reading ISL 3.1.3}
\href{https://docs.google.com/presentation/d/1FuUe3jrFoCwA8XTkoSZBwz8xZ0oGJmpUGOmjWqJJAIc/edit\#slide=id.p}{ISL
Slides}

    \hypertarget{notebook-set-up}{%
\subsection{Notebook Set Up}\label{notebook-set-up}}

    \begin{Verbatim}[commandchars=\\\{\}]
{\color{incolor}In [{\color{incolor}1}]:} \PY{c+c1}{\PYZsh{} general imports}
        \PY{k+kn}{import} \PY{n+nn}{sys}
        \PY{k+kn}{import} \PY{n+nn}{csv}
        \PY{k+kn}{import} \PY{n+nn}{numpy} \PY{k}{as} \PY{n+nn}{np}
        \PY{k+kn}{import} \PY{n+nn}{pandas} \PY{k}{as} \PY{n+nn}{pd}
        \PY{k+kn}{import} \PY{n+nn}{matplotlib}
        \PY{k+kn}{import} \PY{n+nn}{matplotlib}\PY{n+nn}{.}\PY{n+nn}{pyplot} \PY{k}{as} \PY{n+nn}{plt}
        \PY{k+kn}{import} \PY{n+nn}{seaborn} \PY{k}{as} \PY{n+nn}{sns}
        \PY{k+kn}{from} \PY{n+nn}{IPython}\PY{n+nn}{.}\PY{n+nn}{display} \PY{k}{import} \PY{n}{Image}
        
        \PY{c+c1}{\PYZsh{} magic commands}
        \PY{o}{\PYZpc{}}\PY{k}{matplotlib} inline
        \PY{o}{\PYZpc{}}\PY{k}{reload\PYZus{}ext} autoreload
        \PY{o}{\PYZpc{}}\PY{k}{autoreload} 2
\end{Verbatim}


    \begin{Verbatim}[commandchars=\\\{\}]
{\color{incolor}In [{\color{incolor}2}]:} \PY{c+c1}{\PYZsh{} import helper modules}
        \PY{k+kn}{import} \PY{n+nn}{helperFunc}
        \PY{k+kn}{import} \PY{n+nn}{linRegFunc}
        
        \PY{c+c1}{\PYZsh{} OPTIONAL \PYZhy{} uncomment to print helper file docstrings}
        \PY{n+nb}{print}\PY{p}{(}\PY{n}{helperFunc}\PY{o}{.}\PY{n+nv+vm}{\PYZus{}\PYZus{}doc\PYZus{}\PYZus{}}\PY{p}{)}
        \PY{c+c1}{\PYZsh{}print(linRegFunc.\PYZus{}\PYZus{}doc\PYZus{}\PYZus{})}
\end{Verbatim}


    \begin{Verbatim}[commandchars=\\\{\}]

This file contains helper functions for generating, transforming
and plotting 2 dimensional data to use in testing \& for ML demos.

Avaliable functions include:
    augment(X)
    plot2DModels(data, models=[], names = [], title=None)
    plotErrorSurface(data, weight\_grid, loss, title=None)



    \end{Verbatim}

    \hypertarget{a-small-example}{%
\section{A Small Example}\label{a-small-example}}

We'll start with a small example of 5 2-D points:

    \begin{Verbatim}[commandchars=\\\{\}]
{\color{incolor}In [{\color{incolor}3}]:} \PY{o}{\PYZpc{}\PYZpc{}}\PY{k}{writefile} fivePoints.csv
        1,2
        3,4
        5,5
        4,3
        2,3
\end{Verbatim}


    \begin{Verbatim}[commandchars=\\\{\}]
Writing fivePoints.csv

    \end{Verbatim}

    \begin{Verbatim}[commandchars=\\\{\}]
{\color{incolor}In [{\color{incolor}4}]:} \PY{c+c1}{\PYZsh{} load data from file}
        \PY{n}{points} \PY{o}{=} \PY{n}{np}\PY{o}{.}\PY{n}{genfromtxt}\PY{p}{(}\PY{l+s+s2}{\PYZdq{}}\PY{l+s+s2}{fivePoints.csv}\PY{l+s+s2}{\PYZdq{}}\PY{p}{,} \PY{n}{delimiter}\PY{o}{=}\PY{l+s+s1}{\PYZsq{}}\PY{l+s+s1}{,}\PY{l+s+s1}{\PYZsq{}}\PY{p}{)}
\end{Verbatim}


    Here's what they look like next to a the simplest possible linear model:
\$ y = x\$

    \begin{Verbatim}[commandchars=\\\{\}]
{\color{incolor}In [{\color{incolor}5}]:} \PY{c+c1}{\PYZsh{} easy plotting with a helper function}
        \PY{n}{helperFunc}\PY{o}{.}\PY{n}{plot2DModels}\PY{p}{(}\PY{n}{points}\PY{p}{,} \PY{p}{[}\PY{p}{[}\PY{l+m+mi}{1}\PY{p}{,}\PY{l+m+mi}{0}\PY{p}{]}\PY{p}{]}\PY{p}{,}\PY{p}{[}\PY{l+s+s1}{\PYZsq{}}\PY{l+s+s1}{model}\PY{l+s+s1}{\PYZsq{}}\PY{p}{]}\PY{p}{,} \PY{n}{title} \PY{o}{=} \PY{l+s+s1}{\PYZsq{}}\PY{l+s+s1}{Small Example}\PY{l+s+s1}{\PYZsq{}}\PY{p}{)}
\end{Verbatim}


    \begin{center}
    \adjustimage{max size={0.9\linewidth}{0.9\paperheight}}{output_14_0.png}
    \end{center}
    { \hspace*{\fill} \\}
    
    Looks reasonable, but its hard to gauge exactly how good a fit we have
just by looking.

\begin{quote}
\textbf{A TASK FOR YOU:} Fill in the calculations below to compute the
``Training Loss'' for our data. These are easy and intuitive
calculations that you will know from long-ago math classes\ldots{} but
instead of relying on your visual intuition, challenge yourself to think
through these numbers in the context of our matrix equation for loss.
Here it is again for your reference: \begin{equation}\tag{1.3}
f(\boldsymbol{\theta}) = \frac{1}{n}\sum_{i=1}^{n}\left[ \boldsymbol{\theta}^T\cdot\mathbf{x}'_i - y_i\right]^2
\end{equation}
\end{quote}

    \begin{Verbatim}[commandchars=\\\{\}]
{\color{incolor}In [{\color{incolor}6}]:} \PY{o}{!}cat fivePoints.csv
\end{Verbatim}


    \begin{Verbatim}[commandchars=\\\{\}]
1,2
3,4
5,5
4,3
2,3
    \end{Verbatim}

    The parameter vector \(\theta\) for our initial line \(y=x\) is: \$

\begin{bmatrix} ? \ \quad ? \ \end{bmatrix}

\$

The (augmented) data points \(x_j\) are: \$

\begin{bmatrix} ? \\ ? \\ \end{bmatrix}

\$ \$

\begin{bmatrix} ? \\ ? \\ \end{bmatrix}

\$ \$

\begin{bmatrix} ? \\ ? \\ \end{bmatrix}

\$ \$

\begin{bmatrix} ? \\ ? \\ \end{bmatrix}

\$ \$

\begin{bmatrix} ? \\ ? \\ \end{bmatrix}

\$

Our loss calculations will be:

\begin{longtable}[]{@{}cccc@{}}
\toprule
\begin{minipage}[b]{0.08\columnwidth}\centering
\(i\)\strut
\end{minipage} & \begin{minipage}[b]{0.16\columnwidth}\centering
\(y_i\)\strut
\end{minipage} & \begin{minipage}[b]{0.26\columnwidth}\centering
\(\boldsymbol{\theta}^T\cdot\mathbf{x}'_i\)\strut
\end{minipage} & \begin{minipage}[b]{0.39\columnwidth}\centering
\(\left[ \boldsymbol{\theta}^T\cdot\mathbf{x}'_i - y_i\right]^2\)\strut
\end{minipage}\tabularnewline
\midrule
\endhead
\begin{minipage}[t]{0.08\columnwidth}\centering
-\strut
\end{minipage} & \begin{minipage}[t]{0.16\columnwidth}\centering
true y\strut
\end{minipage} & \begin{minipage}[t]{0.26\columnwidth}\centering
predicted y\strut
\end{minipage} & \begin{minipage}[t]{0.39\columnwidth}\centering
squared residual\strut
\end{minipage}\tabularnewline
\begin{minipage}[t]{0.08\columnwidth}\centering
1\strut
\end{minipage} & \begin{minipage}[t]{0.16\columnwidth}\centering
-\strut
\end{minipage} & \begin{minipage}[t]{0.26\columnwidth}\centering
-\strut
\end{minipage} & \begin{minipage}[t]{0.39\columnwidth}\centering
-\strut
\end{minipage}\tabularnewline
\begin{minipage}[t]{0.08\columnwidth}\centering
2\strut
\end{minipage} & \begin{minipage}[t]{0.16\columnwidth}\centering
-\strut
\end{minipage} & \begin{minipage}[t]{0.26\columnwidth}\centering
-\strut
\end{minipage} & \begin{minipage}[t]{0.39\columnwidth}\centering
-\strut
\end{minipage}\tabularnewline
\begin{minipage}[t]{0.08\columnwidth}\centering
3\strut
\end{minipage} & \begin{minipage}[t]{0.16\columnwidth}\centering
-\strut
\end{minipage} & \begin{minipage}[t]{0.26\columnwidth}\centering
-\strut
\end{minipage} & \begin{minipage}[t]{0.39\columnwidth}\centering
-\strut
\end{minipage}\tabularnewline
\begin{minipage}[t]{0.08\columnwidth}\centering
4\strut
\end{minipage} & \begin{minipage}[t]{0.16\columnwidth}\centering
-\strut
\end{minipage} & \begin{minipage}[t]{0.26\columnwidth}\centering
-\strut
\end{minipage} & \begin{minipage}[t]{0.39\columnwidth}\centering
-\strut
\end{minipage}\tabularnewline
\begin{minipage}[t]{0.08\columnwidth}\centering
5\strut
\end{minipage} & \begin{minipage}[t]{0.16\columnwidth}\centering
-\strut
\end{minipage} & \begin{minipage}[t]{0.26\columnwidth}\centering
-\strut
\end{minipage} & \begin{minipage}[t]{0.39\columnwidth}\centering
-\strut
\end{minipage}\tabularnewline
\bottomrule
\end{longtable}

The training loss \(f(\boldsymbol{\theta})\) for this data and these
weights is: \_\_\_\_\_\_\_

    \begin{Verbatim}[commandchars=\\\{\}]
{\color{incolor}In [{\color{incolor}49}]:} \PY{k+kn}{from} \PY{n+nn}{IPython}\PY{n+nn}{.}\PY{n+nn}{display} \PY{k}{import} \PY{n}{Image}
         \PY{n}{Image}\PY{p}{(}\PY{n}{filename}\PY{o}{=}\PY{l+s+s2}{\PYZdq{}}\PY{l+s+s2}{small\PYZus{}example\PYZus{}loss.jpg}\PY{l+s+s2}{\PYZdq{}}\PY{p}{)}
\end{Verbatim}

\texttt{\color{outcolor}Out[{\color{outcolor}49}]:}
    
    \begin{center}
    \adjustimage{max size={0.9\linewidth}{0.9\paperheight}}{output_18_0.jpeg}
    \end{center}
    { \hspace*{\fill} \\}
    

    \begin{Verbatim}[commandchars=\\\{\}]
{\color{incolor}In [{\color{incolor}7}]:} \PY{c+c1}{\PYZsh{} Run this cell to confirm your Hand Calculations}
        \PY{n}{X} \PY{o}{=} \PY{n}{helperFunc}\PY{o}{.}\PY{n}{augment}\PY{p}{(}\PY{n}{points}\PY{p}{)}\PY{p}{[}\PY{p}{:}\PY{p}{,}\PY{p}{:}\PY{o}{\PYZhy{}}\PY{l+m+mi}{1}\PY{p}{]}
        \PY{n}{y} \PY{o}{=} \PY{n}{points}\PY{p}{[}\PY{p}{:}\PY{p}{,}\PY{o}{\PYZhy{}}\PY{l+m+mi}{1}\PY{p}{]}
        \PY{n+nb}{print}\PY{p}{(}\PY{l+s+s2}{\PYZdq{}}\PY{l+s+s2}{Loss:}\PY{l+s+s2}{\PYZdq{}}\PY{p}{,} \PY{n}{linRegFunc}\PY{o}{.}\PY{n}{OLSLoss}\PY{p}{(}\PY{n}{X}\PY{p}{,} \PY{n}{y}\PY{p}{,}\PY{p}{[}\PY{l+m+mi}{1}\PY{p}{,}\PY{l+m+mi}{0}\PY{p}{]}\PY{p}{)}\PY{p}{)}
\end{Verbatim}


    \begin{Verbatim}[commandchars=\\\{\}]
Loss: 0.8

    \end{Verbatim}

    \begin{quote}
\textbf{DISCUSSION QUESTIONS:} * \emph{What parts of this computation
could be parallelized? What, if any, aggregation has to happen at the
end?} * \emph{What key-value format, partitioning, sorting would help?
Could you use a combiner?} * \emph{In addition to the data stream, what
other information would your map or reduce tasks need access to?}
\end{quote}

    \hypertarget{solution}{%
\subsubsection{\textless{}--- SOLUTION
---\textgreater{}}\label{solution}}

\textbf{INSTRUCTOR TALKING POINTS}\\
* \emph{What parts of this computation could be parallelized? What, if
any, aggregation has to happen at the end?} \textgreater{} Everything to
the right of the summation can be calculated on a per row basis. The
aggregation (summation) as well as the final division has to be dne at
the end.\\
* \emph{What key-value format, partitioning, sorting would help? Could
you use a combiner?} \textgreater{} No key or special partitioning is
needed as there is no grouping required. A combiner can be used to sum
local results. Remember to pass the number of rows in the payload to be
able to take the mean at the end.\\
* \emph{In addition to the data stream, what other information would
your map or reduce tasks need access to?} \textgreater{} number of rows.
see above.

    \hypertarget{demo-random-parameter-search.}{%
\subsection{Demo: Random Parameter
Search.}\label{demo-random-parameter-search.}}

    Ok, so we know the model looks ok and we know its loss is \(0.8\) but is
that any good? A naive approach to ``learning'' a Linear Model might be
to randomly generate a few more models and then pick the model with the
lowest loss. Let's try it.

    \begin{Verbatim}[commandchars=\\\{\}]
{\color{incolor}In [{\color{incolor}9}]:} \PY{k+kn}{import} \PY{n+nn}{helperFunc}\PY{o}{,} \PY{n+nn}{linRegFunc}
        
        \PY{c+c1}{\PYZsh{}\PYZsh{}\PYZsh{}\PYZsh{}\PYZsh{}\PYZsh{}\PYZsh{}\PYZsh{}\PYZsh{}\PYZsh{}\PYZsh{}\PYZsh{}\PYZsh{}\PYZsh{}\PYZsh{}\PYZsh{}\PYZsh{}\PYZsh{}\PYZsh{}\PYZsh{} Demo Parameters \PYZsh{}\PYZsh{}\PYZsh{}\PYZsh{}\PYZsh{}\PYZsh{}\PYZsh{}\PYZsh{}\PYZsh{}\PYZsh{}\PYZsh{}\PYZsh{}\PYZsh{}\PYZsh{}\PYZsh{}\PYZsh{}\PYZsh{}\PYZsh{}\PYZsh{}\PYZsh{} }
        \PY{c+c1}{\PYZsh{} TRY CHANGING THESE \PYZam{} SEE HOW IT AFFECTS OUR SEARCH}
        \PY{n}{NUM\PYZus{}MODELS} \PY{o}{=} \PY{l+m+mi}{10}
        \PY{n}{PARAM\PYZus{}RANGE} \PY{o}{=} \PY{p}{[}\PY{o}{\PYZhy{}}\PY{l+m+mi}{5}\PY{p}{,}\PY{l+m+mi}{5}\PY{p}{]}
        
        \PY{c+c1}{\PYZsh{}\PYZsh{}\PYZsh{}\PYZsh{}\PYZsh{}\PYZsh{}\PYZsh{}\PYZsh{}\PYZsh{}\PYZsh{}\PYZsh{}\PYZsh{}\PYZsh{}\PYZsh{}\PYZsh{}\PYZsh{}\PYZsh{}\PYZsh{}\PYZsh{}\PYZsh{} Random Search Demo \PYZsh{}\PYZsh{}\PYZsh{}\PYZsh{}\PYZsh{}\PYZsh{}\PYZsh{}\PYZsh{}\PYZsh{}\PYZsh{}\PYZsh{}\PYZsh{}\PYZsh{}\PYZsh{}\PYZsh{}\PYZsh{}\PYZsh{}\PYZsh{}\PYZsh{}\PYZsh{}}
        \PY{c+c1}{\PYZsh{} Load \PYZam{} pre\PYZhy{}process data}
        \PY{n}{points} \PY{o}{=} \PY{n}{np}\PY{o}{.}\PY{n}{genfromtxt}\PY{p}{(}\PY{l+s+s2}{\PYZdq{}}\PY{l+s+s2}{fivePoints.csv}\PY{l+s+s2}{\PYZdq{}}\PY{p}{,} \PY{n}{delimiter}\PY{o}{=}\PY{l+s+s1}{\PYZsq{}}\PY{l+s+s1}{,}\PY{l+s+s1}{\PYZsq{}}\PY{p}{)}
        \PY{n}{X} \PY{o}{=} \PY{n}{helperFunc}\PY{o}{.}\PY{n}{augment}\PY{p}{(}\PY{n}{points}\PY{p}{)}\PY{p}{[}\PY{p}{:}\PY{p}{,}\PY{p}{:}\PY{l+m+mi}{2}\PY{p}{]}
        \PY{n}{y} \PY{o}{=} \PY{n}{points}\PY{p}{[}\PY{p}{:}\PY{p}{,}\PY{l+m+mi}{1}\PY{p}{]}
        
        \PY{c+c1}{\PYZsh{} \PYZdq{}Training\PYZdq{}}
        \PY{n}{models} \PY{o}{=} \PY{p}{[}\PY{p}{[}\PY{l+m+mi}{0}\PY{p}{,}\PY{l+m+mi}{1}\PY{p}{]}\PY{p}{]}
        \PY{n}{names} \PY{o}{=} \PY{p}{[}\PY{l+s+s2}{\PYZdq{}}\PY{l+s+s2}{INIT \PYZhy{} Loss: 0.8}\PY{l+s+s2}{\PYZdq{}}\PY{p}{]}
        \PY{n}{best} \PY{o}{=} \PY{p}{\PYZob{}}\PY{l+s+s1}{\PYZsq{}}\PY{l+s+s1}{loss}\PY{l+s+s1}{\PYZsq{}}\PY{p}{:}\PY{l+m+mf}{0.8}\PY{p}{,} \PY{l+s+s1}{\PYZsq{}}\PY{l+s+s1}{W}\PY{l+s+s1}{\PYZsq{}}\PY{p}{:} \PY{p}{[}\PY{l+m+mi}{1}\PY{p}{,}\PY{l+m+mi}{0}\PY{p}{]}\PY{p}{\PYZcb{}}
        \PY{k}{for} \PY{n}{idx} \PY{o+ow}{in} \PY{n+nb}{range}\PY{p}{(}\PY{n}{NUM\PYZus{}MODELS}\PY{p}{)}\PY{p}{:}
            \PY{c+c1}{\PYZsh{} initialize a random weight vector w/ values in specified range}
            \PY{n}{W} \PY{o}{=} \PY{n}{np}\PY{o}{.}\PY{n}{random}\PY{o}{.}\PY{n}{uniform}\PY{p}{(}\PY{n}{PARAM\PYZus{}RANGE}\PY{p}{[}\PY{l+m+mi}{0}\PY{p}{]}\PY{p}{,}\PY{n}{PARAM\PYZus{}RANGE}\PY{p}{[}\PY{l+m+mi}{1}\PY{p}{]}\PY{p}{,} \PY{n}{size}\PY{o}{=}\PY{p}{(}\PY{l+m+mi}{2}\PY{p}{)}\PY{p}{)}
            \PY{c+c1}{\PYZsh{} compute loss \PYZam{} store for plotting}
            \PY{n}{loss} \PY{o}{=} \PY{n}{linRegFunc}\PY{o}{.}\PY{n}{OLSLoss}\PY{p}{(}\PY{n}{X}\PY{p}{,} \PY{n}{y}\PY{p}{,} \PY{n}{W}\PY{p}{)}
            \PY{n}{models}\PY{o}{.}\PY{n}{append}\PY{p}{(}\PY{n}{W}\PY{p}{)}
            \PY{n}{names}\PY{o}{.}\PY{n}{append}\PY{p}{(}\PY{l+s+s2}{\PYZdq{}}\PY{l+s+s2}{model}\PY{l+s+si}{\PYZpc{}s}\PY{l+s+s2}{ \PYZhy{} Loss: }\PY{l+s+si}{\PYZpc{}.2f}\PY{l+s+s2}{\PYZdq{}} \PY{o}{\PYZpc{}} \PY{p}{(}\PY{n}{idx}\PY{p}{,} \PY{n}{loss}\PY{p}{)}\PY{p}{)}
            \PY{c+c1}{\PYZsh{} track best model}
            \PY{k}{if} \PY{n}{loss} \PY{o}{\PYZlt{}} \PY{n}{best}\PY{p}{[}\PY{l+s+s1}{\PYZsq{}}\PY{l+s+s1}{loss}\PY{l+s+s1}{\PYZsq{}}\PY{p}{]}\PY{p}{:}
                \PY{n}{best}\PY{p}{[}\PY{l+s+s1}{\PYZsq{}}\PY{l+s+s1}{loss}\PY{l+s+s1}{\PYZsq{}}\PY{p}{]} \PY{o}{=} \PY{n}{loss}
                \PY{n}{best}\PY{p}{[}\PY{l+s+s1}{\PYZsq{}}\PY{l+s+s1}{W}\PY{l+s+s1}{\PYZsq{}}\PY{p}{]} \PY{o}{=} \PY{n}{W}
                
        \PY{c+c1}{\PYZsh{} Display Results}
        \PY{n+nb}{print}\PY{p}{(}\PY{n}{f}\PY{l+s+s2}{\PYZdq{}}\PY{l+s+s2}{Best Random Model: }\PY{l+s+si}{\PYZob{}best[\PYZsq{}W\PYZsq{}]\PYZcb{}}\PY{l+s+s2}{, Loss: }\PY{l+s+si}{\PYZob{}best[\PYZsq{}loss\PYZsq{}]\PYZcb{}}\PY{l+s+s2}{\PYZdq{}}\PY{p}{)}
        \PY{n}{helperFunc}\PY{o}{.}\PY{n}{plot2DModels}\PY{p}{(}\PY{n}{points}\PY{p}{,} \PY{n}{models}\PY{p}{,} \PY{n}{names}\PY{p}{,} \PY{l+s+s2}{\PYZdq{}}\PY{l+s+s2}{A Random Approach.}\PY{l+s+s2}{\PYZdq{}}\PY{p}{)}
\end{Verbatim}


    \begin{Verbatim}[commandchars=\\\{\}]
Best Random Model: [1, 0], Loss: 0.8

    \end{Verbatim}

    \begin{center}
    \adjustimage{max size={0.9\linewidth}{0.9\paperheight}}{output_24_1.png}
    \end{center}
    { \hspace*{\fill} \\}
    
    So, that was pretty poor. One idea would be to run a lot more
iterations.

\begin{quote}
\textbf{DISCUSSION QUESTION:} * \emph{To what extent could
parallelization help us redeem this approach? What exactly would you
parallelize?}
\end{quote}

    \hypertarget{solution}{%
\subsubsection{\textless{}--- SOLUTION
---\textgreater{}}\label{solution}}

\textbf{INSTRUCTOR TALKING POINTS} * \emph{To what extent could
parallelization help us redeem this approach? What exactly would you
parallelize?} \textgreater{} While parallelization could help us train a
lot more models in the same amount of time, we'd have no real guarantee
that we'd get better results in exchange for our efforts because of the
`randomness' of what models we try.

    \hypertarget{demo-systematic-brute-force.}{%
\subsection{Demo: Systematic Brute
Force.}\label{demo-systematic-brute-force.}}

    For obvious reasons a more systematic approach is desirable. Instead of
randomly guessing, let's use what we know to search an appropriate
section of the the model space.

We can tell from the data that the linear model should probably have a
fairly shallow positive slope and a positive intercept between 0 and 2.
So lets initialize every possible combination of weights in that range
up to a granularity of, say \(0.2\), and compute the loss for each one.

    \begin{Verbatim}[commandchars=\\\{\}]
{\color{incolor}In [{\color{incolor}10}]:} \PY{k+kn}{import} \PY{n+nn}{helperFunc}\PY{o}{,} \PY{n+nn}{linRegFunc}
         
         \PY{c+c1}{\PYZsh{}\PYZsh{}\PYZsh{}\PYZsh{}\PYZsh{}\PYZsh{}\PYZsh{}\PYZsh{}\PYZsh{}\PYZsh{}\PYZsh{}\PYZsh{}\PYZsh{}\PYZsh{}\PYZsh{}\PYZsh{}\PYZsh{}\PYZsh{}\PYZsh{}\PYZsh{} Demo Parameters \PYZsh{}\PYZsh{}\PYZsh{}\PYZsh{}\PYZsh{}\PYZsh{}\PYZsh{}\PYZsh{}\PYZsh{}\PYZsh{}\PYZsh{}\PYZsh{}\PYZsh{}\PYZsh{}\PYZsh{}\PYZsh{}\PYZsh{}\PYZsh{}\PYZsh{}\PYZsh{} }
         \PY{c+c1}{\PYZsh{} TRY CHANGING THESE \PYZam{} SEE HOW IT AFFECTS OUR SEARCH}
         \PY{n}{W0\PYZus{}MIN} \PY{o}{=} \PY{l+m+mi}{0}
         \PY{n}{W0\PYZus{}MAX} \PY{o}{=} \PY{l+m+mi}{2}
         \PY{n}{W0\PYZus{}STEP} \PY{o}{=} \PY{l+m+mf}{0.2}
         
         \PY{n}{W1\PYZus{}MIN} \PY{o}{=} \PY{l+m+mi}{0}
         \PY{n}{W1\PYZus{}MAX} \PY{o}{=} \PY{l+m+mi}{2}
         \PY{n}{W1\PYZus{}STEP} \PY{o}{=} \PY{l+m+mf}{0.2}
         
         \PY{c+c1}{\PYZsh{}\PYZsh{}\PYZsh{}\PYZsh{}\PYZsh{}\PYZsh{}\PYZsh{}\PYZsh{}\PYZsh{}\PYZsh{}\PYZsh{}\PYZsh{}\PYZsh{}\PYZsh{}\PYZsh{}\PYZsh{}\PYZsh{}\PYZsh{}\PYZsh{}\PYZsh{} Grid Search Demo \PYZsh{}\PYZsh{}\PYZsh{}\PYZsh{}\PYZsh{}\PYZsh{}\PYZsh{}\PYZsh{}\PYZsh{}\PYZsh{}\PYZsh{}\PYZsh{}\PYZsh{}\PYZsh{}\PYZsh{}\PYZsh{}\PYZsh{}\PYZsh{}\PYZsh{}\PYZsh{} }
         \PY{c+c1}{\PYZsh{}\PYZsh{}\PYZsh{} Load \PYZam{} Pre\PYZhy{}process Data}
         \PY{n}{points} \PY{o}{=} \PY{n}{np}\PY{o}{.}\PY{n}{genfromtxt}\PY{p}{(}\PY{l+s+s2}{\PYZdq{}}\PY{l+s+s2}{fivePoints.csv}\PY{l+s+s2}{\PYZdq{}}\PY{p}{,} \PY{n}{delimiter}\PY{o}{=}\PY{l+s+s1}{\PYZsq{}}\PY{l+s+s1}{,}\PY{l+s+s1}{\PYZsq{}}\PY{p}{)}
         \PY{n}{X} \PY{o}{=} \PY{n}{helperFunc}\PY{o}{.}\PY{n}{augment}\PY{p}{(}\PY{n}{points}\PY{p}{)}\PY{p}{[}\PY{p}{:}\PY{p}{,}\PY{p}{:}\PY{l+m+mi}{2}\PY{p}{]}
         \PY{n}{y} \PY{o}{=} \PY{n}{points}\PY{p}{[}\PY{p}{:}\PY{p}{,}\PY{l+m+mi}{1}\PY{p}{]}
         
         \PY{c+c1}{\PYZsh{}\PYZsh{}\PYZsh{} \PYZdq{}Training\PYZdq{} }
         \PY{c+c1}{\PYZsh{} create a model for each point in our grid}
         \PY{n}{grid} \PY{o}{=} \PY{n}{np}\PY{o}{.}\PY{n}{mgrid}\PY{p}{[}\PY{n}{W0\PYZus{}MIN}\PY{p}{:}\PY{n}{W0\PYZus{}MAX}\PY{p}{:}\PY{n}{W0\PYZus{}STEP}\PY{p}{,}\PY{n}{W1\PYZus{}MIN}\PY{p}{:}\PY{n}{W1\PYZus{}MAX}\PY{p}{:}\PY{n}{W1\PYZus{}STEP}\PY{p}{]}
         \PY{n}{size} \PY{o}{=} \PY{n+nb}{int}\PY{p}{(}\PY{n}{np}\PY{o}{.}\PY{n}{product}\PY{p}{(}\PY{n}{grid}\PY{o}{.}\PY{n}{shape}\PY{p}{)}\PY{o}{/}\PY{l+m+mi}{2}\PY{p}{)}
         \PY{n}{models} \PY{o}{=} \PY{n}{grid}\PY{o}{.}\PY{n}{reshape}\PY{p}{(}\PY{l+m+mi}{2}\PY{p}{,}\PY{n}{size}\PY{p}{)}\PY{o}{.}\PY{n}{T}
         \PY{c+c1}{\PYZsh{} compute loss for each model}
         \PY{n}{loss} \PY{o}{=} \PY{p}{[}\PY{p}{]}
         \PY{k}{for} \PY{n}{W} \PY{o+ow}{in} \PY{n}{models}\PY{p}{:}
             \PY{n}{loss}\PY{o}{.}\PY{n}{append}\PY{p}{(}\PY{n}{linRegFunc}\PY{o}{.}\PY{n}{OLSLoss}\PY{p}{(}\PY{n}{X}\PY{p}{,}\PY{n}{y}\PY{p}{,}\PY{n}{W}\PY{p}{)}\PY{p}{)}
             
         \PY{c+c1}{\PYZsh{}\PYZsh{}\PYZsh{} Display Results}
         \PY{n+nb}{print}\PY{p}{(}\PY{n}{f}\PY{l+s+s2}{\PYZdq{}}\PY{l+s+s2}{Searched }\PY{l+s+si}{\PYZob{}size\PYZcb{}}\PY{l+s+s2}{ models...}\PY{l+s+s2}{\PYZdq{}}\PY{p}{)}
         \PY{n+nb}{print}\PY{p}{(}\PY{n}{f}\PY{l+s+s2}{\PYZdq{}}\PY{l+s+s2}{Best model: }\PY{l+s+si}{\PYZob{}models[np.argmin(loss)]\PYZcb{}}\PY{l+s+s2}{, Loss: }\PY{l+s+s2}{\PYZob{}}\PY{l+s+s2}{min(loss)\PYZcb{}}\PY{l+s+s2}{\PYZdq{}}\PY{p}{)}
         \PY{n}{helperFunc}\PY{o}{.}\PY{n}{plotErrorSurface}\PY{p}{(}\PY{n}{points}\PY{p}{,}\PY{n}{models}\PY{p}{,}\PY{n}{loss}\PY{p}{)}
\end{Verbatim}


    \begin{Verbatim}[commandchars=\\\{\}]
Searched 100 models{\ldots}
Best model: [0.6 1.6], Loss: 0.31999999999999984

    \end{Verbatim}

    \begin{center}
    \adjustimage{max size={0.9\linewidth}{0.9\paperheight}}{output_29_1.png}
    \end{center}
    { \hspace*{\fill} \\}
    
    \begin{quote}
\textbf{DISCUSSION QUESTIONS:} * \emph{When we think about scaling up,
is this still a better approach than guessing? How could it be
parallelized?} * \emph{What would change about this approach if we had
higher dimension data?} * \emph{In practice, when we're training Linear
Models why don't we just look at the error surface and identify the
lowest point?} * \emph{What about if we're training other kinds of
models?}
\end{quote}

    \hypertarget{solution}{%
\subsubsection{\textless{}--- SOLUTION
---\textgreater{}}\label{solution}}

\textbf{INSTRUCTOR TALKING POINTS}\\
* \emph{When we think about scaling up, is this still a better approach
than guessing? How could it be parallelized?} \textgreater{} Yes, we can
at least methodically and incrementally improve the solution. The same
parralellization methods still apply. * \emph{What would change about
this approach if we had higher dimension data?} \textgreater{} With more
and more dimensions there would be an exponential number of models to
search even a small grid. * \emph{In practice, when we're training
Linear Models why don't we just look at the error surface and identify
the lowest point?} \textgreater{} We do not have access to the error
surface until we've computed the loss for every possible comnination of
parameters. Doing so is computationally challenging. There are better
methods at arriving at the optimal solution.

    \hypertarget{parameter-space-gradients-and-convexity}{%
\section{Parameter Space, Gradients, and
Convexity}\label{parameter-space-gradients-and-convexity}}

As suggested by the systematic search demo, when we train parametric
models we tend to switch back and forth between two different ways of
visualizing our goal.

    \begin{Verbatim}[commandchars=\\\{\}]
{\color{incolor}In [{\color{incolor}11}]:} \PY{k+kn}{from} \PY{n+nn}{IPython}\PY{n+nn}{.}\PY{n+nn}{display} \PY{k}{import} \PY{n}{HTML}
         \PY{n}{HTML}\PY{p}{(}\PY{l+s+s1}{\PYZsq{}}\PY{l+s+s1}{\PYZlt{}img src=}\PY{l+s+s1}{\PYZdq{}}\PY{l+s+s1}{./GD\PYZus{}gif/Gradient\PYZus{}Descent.gif}\PY{l+s+s1}{\PYZdq{}}\PY{l+s+s1}{\PYZgt{}}\PY{l+s+s1}{\PYZsq{}}\PY{p}{)}
\end{Verbatim}


\begin{Verbatim}[commandchars=\\\{\}]
{\color{outcolor}Out[{\color{outcolor}11}]:} <IPython.core.display.HTML object>
\end{Verbatim}
            
    \begin{itemize}
\tightlist
\item
  When we look at a model next to our data represented in the Problem
  Domain Space, it is natural to think about loss as a measure of
  \emph{\textbf{how far off the data are from our model}}. In other
  words, this visual suggests loss is a function of the training data
  \(X\).
\item
  By contrast, looking at an error surface plotted in Model Parameter
  Space, we intuitively see loss as an indicator of \emph{\textbf{how
  far off the current model is from the optimal model}}. In other words,
  this view helps us think of loss as a function of the parameters
  \(\theta\).
\end{itemize}

Of course in one sense, this distinction is just a matter of sematics.
As we saw in equations 1.2, 1.3 and 1.4, MSE loss depends on \emph{both}
the data and the parameters. However, in the context of `inventing' ways
to train a model, this distinction is a useful one. If we think of the
data as fixed and focus on how loss varies \emph{with respect to the
parameters}, then we can take advantage of a little theory to speed up
our search for the optimal parameters.

    \hypertarget{optimization-theory-a-short-digression}{%
\subsubsection{Optimization Theory \ldots{} a short
digression}\label{optimization-theory-a-short-digression}}

Calculus gives us the simple solution to optimizing a real function. The
\textbf{First Order Conditions} (a.k.a. `first derivative rule') says
that the maximum or minimum of an unconstrained function must occur at a
point where the first derivative = 0. In higher dimensions we extend
this rule to talk about a \textbf{gradient} vector of partial
derivatives which all must equal 0.

When the first order partial derivatives are equal to zero, then we know
we are at a local maximum or minimum of the real function. But which one
is it? In order to tell, we must take the second derivatives of the real
function. If the second derivatives are positive at that point, then we
know we are at a minimum. If the second derivatives are negative, then
we know we are at a maximum. These are the \textbf{second order
conditions.}

\textbf{Convex Optimization} is the lucky case where we know that the
second derivatives never change sign. There are lots of complicated loss
functions for which we can't easily visualize the error surface but for
which we \emph{can} prove mathematically that this 2nd order condition
is met. If this is the case, then we can think of the suface as
\emph{always curving up} or \emph{always curving down} which guarantees
that any minimum we reach will be an absolute minimum. More powerfully
still, this result can be shown to \emph{also} apply to a class of
``pseudo-convex'' functions - functions whose second derivative might
not be well defined, but satisfy certain conditions that allow us to
guarantee convergence.

    \begin{quote}
\textbf{DSICUSSION QUESTIONS:} * \emph{In the case of Linear Regression
performed on data \(X \in \mathbb{R}^m\), how many dimensions does the
gradient vector have? What do each of the values in this vector
represent visually?} * \emph{If we are systematically searching the
parameter space for a lowest point, why might it be useful to know that
our loss function is convex?} * \emph{In general (i.e.~beyond Linear
Regression) if finding the ideal parameters \(\theta\), is as simple as
solving the equation \(f'(\theta)=0\), why don't we always train our
models by solving that equation?} * \emph{Condider the loss curves
illustrated below -- do these illustrations represent problem space or
parameter space? which ones are convex?}
\end{quote}

    Sources:
\href{http://fa.bianp.net/blog/2014/surrogate-loss-functions-in-machine-learning/}{first
image} \textbar{}
\href{https://blog.algorithmia.com/introduction-to-loss-functions/}{second
image} \textbar{}
\href{https://fr.mathworks.com/help/gads/example-finding-the-minimum-of-a-function-using-the-gps-algorithm.html}{third
image}

    \hypertarget{solution}{%
\subsubsection{\textless{}--- SOLUTION
---\textgreater{}}\label{solution}}

\textbf{INSTRUCTOR TALKING POINTS} * \emph{In the case of Linear
Regression performed on data \(X \in \mathbb{R}^m\), how many dimensions
does the gradient vector have? What do each of the values in this vector
represent visually?} \textgreater{} There are \(m\) dimentions each of
which can be thought of as an axis - it is difficult to visualize when
\(m > 3\) * \emph{If we are systematically searching the parameter space
for a lowest point, why might it be useful to know that our loss
function is convex?} \textgreater{} If our loss function was not convex,
we might get stuck in a local minimum before finding the optimal
solution. * \emph{In general (i.e.~beyond Linear Regression) if finding
the ideal parameters \(\theta\), is as simple as solving the equation
\(f'(\theta)=0\), why don't we always train our models by solving that
equation?} \textgreater{} Not all functioins are diferentiable. In
addition, it becomes computationaly difficult in high dimensional spaces
- specifically, it is difficult to invert large matrices.

    \hypertarget{demo-gradient-descent}{%
\subsection{Demo: Gradient Descent}\label{demo-gradient-descent}}

    To take advantage of these lessons from Optimization Theory, we'll start
by taking the derivative of the loss function with respect to the
parameters \(\boldsymbol{\theta}\). Recall the matrix formulation of our
loss function:

\begin{equation}\tag{1.3}
f(\boldsymbol{\theta}) = \frac{1}{n}\sum_{i=1}^{n}\left[ \boldsymbol{\theta}^T\cdot\mathbf{x}'_i - y_i\right]^2
\end{equation}

We can apply the sum and chain derivation rules to compute the gradient:

\begin{equation}\tag{3.1}
\nabla_{\boldsymbol{\theta}} f(\boldsymbol{\theta}) = \frac{2}{n}\,\sum_{i=1}^{n}\left[ \boldsymbol{\theta}^T\cdot\mathbf{x}'_i - y_i\right] \cdot \mathbf{x}'_i
\end{equation}

We \emph{could} now set this equation equal to \(0\) and then solve for
\(\boldsymbol{\theta}\)\ldots{} but it turns out that this
\textbf{closed form solution} can be computationally challenging in
higher dimensions. It also turns out that a simple approximation
technique will work almost as well.

The strategy of \textbf{Gradient Descent} is to start somewhere random
in the Model Parameter Space and then move down the error surface to
find a minimum point with the optimal parameters for our training data.
Its ingeniousness is that we can do this without actually knowing the
full shape of the error surface. Think of it like walking down a hill
while blindfolded. You test each direction to see which way is down,
then take a little step in that direction and repeat the process until
you can't feel any more `down' to go. The `size' of our steps is
controled by a hyperparameter, \(\alpha\), the \textbf{learning rate}.
The whole process can be summarized in 3 steps: 1. Initialize the
parameters \(\theta\). 2. Compute the gradient
\(\nabla_{\boldsymbol{\theta}} f(\boldsymbol{\theta})\). 3. Update the
parameters: \$\theta\emph{\{\text{new}\} = \theta}\{\text{old}\} -
\alpha \cdot \nabla\_\{\boldsymbol{\theta}\} f(\boldsymbol{\theta}) \$

We repeat these steps until we reach a stopping criteria.

    \begin{quote}
\textbf{A TASK FOR YOU:} Compute one Gradent Descent update step for the
small example from Part 2. Recall that our initial parameters were:
\[ \boldsymbol{\theta} = \begin{bmatrix} 1 \ \quad 0 \ \end{bmatrix}\]\\
For your convenience the augmented input data vectors are already
entered in the table below:
\end{quote}

Hand Calculations:

\begin{longtable}[]{@{}cccc@{}}
\toprule
\begin{minipage}[b]{0.09\columnwidth}\centering
\(x_j '\)\strut
\end{minipage} & \begin{minipage}[b]{0.11\columnwidth}\centering
\(y_j\)\strut
\end{minipage} & \begin{minipage}[b]{0.28\columnwidth}\centering
\(\boldsymbol{\theta}\cdot\mathbf{x}'_j\)\strut
\end{minipage} & \begin{minipage}[b]{0.40\columnwidth}\centering
\(\left[ \boldsymbol{\theta}^T\cdot\mathbf{x}'_j - y_j\right]\cdot\mathbf{x}'_j\)\strut
\end{minipage}\tabularnewline
\midrule
\endhead
\begin{minipage}[t]{0.09\columnwidth}\centering
input\strut
\end{minipage} & \begin{minipage}[t]{0.11\columnwidth}\centering
true y\strut
\end{minipage} & \begin{minipage}[t]{0.28\columnwidth}\centering
predicted y\strut
\end{minipage} & \begin{minipage}[t]{0.40\columnwidth}\centering
gradient component for \(x_j\)\strut
\end{minipage}\tabularnewline
\bottomrule
\end{longtable}

\textbar{} \$

\begin{bmatrix} 1 \\ 1 \\ \end{bmatrix}

\$ \textbar{} 2 \textbar{} \textbar{}\\
\textbar{} \$

\begin{bmatrix} 2 \\ 1 \\ \end{bmatrix}

\$ \textbar{} 3 \textbar{} \textbar{}\\
\textbar{} \$

\begin{bmatrix} 3 \\ 1 \\ \end{bmatrix}

\$ \textbar{} 4 \textbar{} \textbar{}\\
\textbar{} \$

\begin{bmatrix} 4 \\ 1 \\ \end{bmatrix}

\$ \textbar{} 3 \textbar{} \textbar{}\\
\textbar{} \$

\begin{bmatrix} 5 \\ 1 \\ \end{bmatrix}

\$ \textbar{} 5 \textbar{} \textbar{}

    The gradient \(\nabla_{\boldsymbol{\theta}} f(\boldsymbol{\theta})\) for
this data and these weights is: {[}-0.8, -0.8{]}

If \(\alpha = 0.1\) the update for this step will be: {[}-0.08 -0.08{]}

The new parameters will be \(\theta_{\text{new}}=\) {[}1.08, 0.08{]}

    \begin{Verbatim}[commandchars=\\\{\}]
{\color{incolor}In [{\color{incolor}12}]:} \PY{n}{Image}\PY{p}{(}\PY{n}{filename}\PY{o}{=}\PY{l+s+s2}{\PYZdq{}}\PY{l+s+s2}{small\PYZus{}example\PYZus{}gradient.jpeg}\PY{l+s+s2}{\PYZdq{}}\PY{p}{)}
\end{Verbatim}

\texttt{\color{outcolor}Out[{\color{outcolor}12}]:}
    
    \begin{center}
    \adjustimage{max size={0.9\linewidth}{0.9\paperheight}}{output_43_0.jpeg}
    \end{center}
    { \hspace*{\fill} \\}
    

    \begin{quote}
\textbf{DISCUSSION QUESTIONS:}\\
* \emph{How would you go about parallelizing this calculation? What
would the mapper do, what would the reducers do? What key-value
structure, sorting, partitioning, etc would you use?} * \emph{How do the
computational demands of performing GD compare to the task of computing
the loss?}
\end{quote}

    \hypertarget{solution}{%
\subsubsection{\textless{}--- SOLUTION
---\textgreater{}}\label{solution}}

\textbf{INSTRUCTOR TALKING POINTS} * \emph{How would you go about
parallelizing this calculation? What would the mapper do, what would the
reducers do? What key-value structure, sorting, partitioning, etc would
you use?} \textgreater{} For each row of data the mappers would output
the result of the equation to the right of the summand. Since we are not
grouping anything, no key is needed, and no particular partitioning
scheme is necessary. If using combiners, one would need to ensure that
the number of rows is passed along in the payload to the reducers. The
reducers would sum the incomning values, and a final reducer would
compute the mean. * \emph{How do the computational demands of performing
GD compare to the task of computing the loss?} \textgreater{} Computing
the loss is a component of computing the gradient. In term of algorithm
complexity they are almost the same.

    \textbf{Run this demo to confirm your hand calculations \& examine a few
more GD steps.}

    \begin{Verbatim}[commandchars=\\\{\}]
{\color{incolor}In [{\color{incolor}11}]:} \PY{k+kn}{import} \PY{n+nn}{numpy} \PY{k}{as} \PY{n+nn}{np}
         \PY{k+kn}{import} \PY{n+nn}{helperFunc}\PY{o}{,} \PY{n+nn}{linRegFunc}
         
         \PY{c+c1}{\PYZsh{}\PYZsh{}\PYZsh{}\PYZsh{}\PYZsh{}\PYZsh{}\PYZsh{}\PYZsh{}\PYZsh{}\PYZsh{}\PYZsh{}\PYZsh{}\PYZsh{}\PYZsh{}\PYZsh{}\PYZsh{}\PYZsh{}\PYZsh{}\PYZsh{}\PYZsh{} Demo Parameters \PYZsh{}\PYZsh{}\PYZsh{}\PYZsh{}\PYZsh{}\PYZsh{}\PYZsh{}\PYZsh{}\PYZsh{}\PYZsh{}\PYZsh{}\PYZsh{}\PYZsh{}\PYZsh{}\PYZsh{}\PYZsh{}\PYZsh{}\PYZsh{}\PYZsh{}\PYZsh{} }
         \PY{c+c1}{\PYZsh{} TRY CHANGING THESE \PYZam{} SEE HOW IT AFFECTS OUR SEARCH}
         \PY{n}{N\PYZus{}STEPS} \PY{o}{=} \PY{l+m+mi}{5}
         \PY{n}{LEARNING\PYZus{}RATE} \PY{o}{=} \PY{l+m+mf}{0.1}
         \PY{n}{ORIGINAL\PYZus{}MODEL} \PY{o}{=} \PY{p}{[}\PY{l+m+mi}{1}\PY{p}{,}\PY{l+m+mi}{0}\PY{p}{]}
         \PY{n}{SHOW\PYZus{}CONTOURS} \PY{o}{=} \PY{k+kc}{True}
         
         \PY{c+c1}{\PYZsh{}\PYZsh{}\PYZsh{}\PYZsh{}\PYZsh{}\PYZsh{}\PYZsh{}\PYZsh{}\PYZsh{}\PYZsh{}\PYZsh{}\PYZsh{}\PYZsh{}\PYZsh{}\PYZsh{}\PYZsh{}\PYZsh{}\PYZsh{}\PYZsh{} Gradient Update Demo \PYZsh{}\PYZsh{}\PYZsh{}\PYZsh{}\PYZsh{}\PYZsh{}\PYZsh{}\PYZsh{}\PYZsh{}\PYZsh{}\PYZsh{}\PYZsh{}\PYZsh{}\PYZsh{}\PYZsh{}\PYZsh{}\PYZsh{}\PYZsh{}\PYZsh{}\PYZsh{} }
         \PY{c+c1}{\PYZsh{}\PYZsh{}\PYZsh{} Load \PYZam{} Pre\PYZhy{}process Data}
         \PY{n}{points} \PY{o}{=} \PY{n}{np}\PY{o}{.}\PY{n}{genfromtxt}\PY{p}{(}\PY{l+s+s2}{\PYZdq{}}\PY{l+s+s2}{fivePoints.csv}\PY{l+s+s2}{\PYZdq{}}\PY{p}{,} \PY{n}{delimiter}\PY{o}{=}\PY{l+s+s1}{\PYZsq{}}\PY{l+s+s1}{,}\PY{l+s+s1}{\PYZsq{}}\PY{p}{)}
         \PY{n}{X} \PY{o}{=} \PY{n}{helperFunc}\PY{o}{.}\PY{n}{augment}\PY{p}{(}\PY{n}{points}\PY{p}{)}\PY{p}{[}\PY{p}{:}\PY{p}{,}\PY{p}{:}\PY{l+m+mi}{2}\PY{p}{]}
         \PY{n}{y} \PY{o}{=} \PY{n}{points}\PY{p}{[}\PY{p}{:}\PY{p}{,}\PY{l+m+mi}{1}\PY{p}{]}
         
         \PY{c+c1}{\PYZsh{}\PYZsh{}\PYZsh{} Perform GD Update \PYZam{} save intermediate model performance}
         \PY{n}{models}\PY{p}{,} \PY{n}{loss} \PY{o}{=} \PY{n}{linRegFunc}\PY{o}{.}\PY{n}{GDUpdate}\PY{p}{(}\PY{n}{X}\PY{p}{,} \PY{n}{y}\PY{p}{,} \PY{n}{N\PYZus{}STEPS}\PY{p}{,}
                                            \PY{n}{ORIGINAL\PYZus{}MODEL}\PY{p}{,} 
                                            \PY{n}{LEARNING\PYZus{}RATE}\PY{p}{,} 
                                            \PY{n}{verbose} \PY{o}{=} \PY{k+kc}{True}\PY{p}{)}
         
         \PY{c+c1}{\PYZsh{}\PYZsh{}\PYZsh{} Display Results}
         \PY{n+nb}{print}\PY{p}{(}\PY{n}{f}\PY{l+s+s2}{\PYZdq{}}\PY{l+s+se}{\PYZbs{}n}\PY{l+s+s2}{Searched }\PY{l+s+s2}{\PYZob{}}\PY{l+s+s2}{len(models)\PYZcb{} models...}\PY{l+s+s2}{\PYZdq{}}\PY{p}{)}
         \PY{n+nb}{print}\PY{p}{(}\PY{n}{f}\PY{l+s+s2}{\PYZdq{}}\PY{l+s+s2}{Best model: }\PY{l+s+si}{\PYZob{}models[np.argmin(loss)]\PYZcb{}}\PY{l+s+s2}{, Loss: }\PY{l+s+si}{\PYZob{}loss[np.argmin(loss)]\PYZcb{}}\PY{l+s+s2}{\PYZdq{}}\PY{p}{)}
         \PY{n}{linRegFunc}\PY{o}{.}\PY{n}{plotGDProgress}\PY{p}{(}\PY{n}{points}\PY{p}{,} \PY{n}{models}\PY{p}{,} \PY{n}{loss}\PY{p}{,}
                                   \PY{n}{show\PYZus{}contours} \PY{o}{=} \PY{n}{SHOW\PYZus{}CONTOURS}\PY{p}{)}
\end{Verbatim}


    \begin{Verbatim}[commandchars=\\\{\}]
Model 0: [1.00, 0.00]
Loss: 0.8
     >>> gradient: [-0.8 -0.8]
     >>> update: [-0.08 -0.08]
Model 1: [1.08, 0.08]
Loss: 0.7872
     >>> gradient: [ 1.44 -0.16]
     >>> update: [ 0.144 -0.016]
Model 2: [0.94, 0.10]
Loss: 0.7918080000000005
     >>> gradient: [-1.632 -0.992]
     >>> update: [-0.1632 -0.0992]
Model 3: [1.10, 0.20]
Loss: 0.82701312
     >>> gradient: [2.5536 0.1856]
     >>> update: [0.25536 0.01856]
Model 4: [0.84, 0.18]
Loss: 0.9175584768000005
     >>> gradient: [-3.17568 -1.38368]
     >>> update: [-0.317568 -0.138368]
Model 5: [1.16, 0.32]
Loss: 1.1097440747520002

Searched 6 models{\ldots}
Best model: [1.08 0.08], Loss: 0.7872

    \end{Verbatim}

    \begin{center}
    \adjustimage{max size={0.9\linewidth}{0.9\paperheight}}{output_47_1.png}
    \end{center}
    { \hspace*{\fill} \\}
    
    \begin{quote}
\textbf{DISCUSSION QUESTIONS:}\\
* \emph{Look closely at the loss for each model, what problem do you
notice?} * \emph{Use the Model Parameter Space view to explain why this
problem might be occurring.} \textbf{HINT:} Try
\texttt{SHOW\_CONTOURS\ =\ True}. \emph{Based upon your insights,
propose a solution to this problem.} * \emph{When performing GD `in the
wild' will we be able to visualize the error surface (eg. using contour
lines, heatmaps or 3D plots)?}
\end{quote}

    \hypertarget{solution}{%
\subsubsection{\textless{}--- SOLUTION
---\textgreater{}}\label{solution}}

\textbf{INSTRUCTOR TALKING POINTS} * \emph{Look closely at the loss for
each model, what problem do you notice?} \textgreater{} The loss is
growing * \emph{Use the Model Parameter Space view to explain why this
problem might be occurring.} \textbf{HINT:} Try
\texttt{SHOW\_CONTOURS\ =\ True}. \emph{Based upon your insights,
propose a solution to this problem.} \textgreater{} We could reduce the
learning rate to prevent the parameter updates from being too large. We
may also consider increasing the number of iterations. * \emph{When
performing GD `in the wild' will we be able to visualize the error
surface (eg. using contour lines, heatmaps or 3D plots)?} \textgreater{}
We would not be able to visualize anything beyond 3 dimensions.

    \hypertarget{demo-stochastic-gradient-descent}{%
\subsection{Demo : Stochastic Gradient
Descent}\label{demo-stochastic-gradient-descent}}

In full Gradient Descent (what we did above) we do a descent step only
after the calculation of the gradient over the whole set of data. That
means we only update the weight vector once each \textbf{epoch} (pass
over the data) thus making one small but ``good'' step towards the
minimum. However since gradient descent is an interative algorithm that
requires many updates to find the minimum, with large datasets, waiting
to process every record before performing an update can result in a slow
and computationaly costly training process.

The alternatives are: 1. \textbf{Stochastic GD} -- compute the gradient
\emph{with respect to a single point at a time} and update the entire
weight vector after each record. By the time we have seen the whole data
set, we will have made N (num of observations), perhaps ``not so good'',
steps with a general trend towards the minimum. SGD will ``zig-zag''
towards the minimum and eventually oscillate around the minimum but
never converge. The advantage of SGD is that we can make progress at
every example - if the data is very large, we may only need 1 pass over
the whole dataset. 2. \textbf{Mini-batch GD} -- compute the gradient
\emph{with respect to a small \textbf{batch} (size of \(B\)) of points
at a time} and update the entire weight vector after each batch. If we
are smart about shuffling the data, this can reduce the ``zig-zaging''
because the points in a batch will temper each other's influence. This
is especially advantageous for noisy data where a single point might
result in a gradient update that is dramatically in the wrong direction
for the rest of the data. For this reason, MBGD can potentially finish
even faster than SGD. However MBGD is seldom used because finding the
right hyper-parameter b is a pain (usually \(B\) is not too big \(-\)
from 32 to 256).

Other than the denominator in front, the loss function for SGD/MBGD
should look very familiar (note that SGD is basically just the special
case where \(B = 1\)):

\begin{equation}\tag{3.2}
\nabla f(\boldsymbol{\theta}) \approx \nabla_{\text{batch}\,\,} f(\boldsymbol{\theta}) = \frac{2}{B}\sum_{i=1}^{B}\left(\boldsymbol{\theta}^T \cdot\mathbf{x}'_{a_i} - y_{a_i}\right)\cdot \mathbf{x}'_{a_i}
\end{equation}

where \(a_i\) is an array of indices of objects which are in this batch.
After obtaining this gradient we do a descent step in this approximate
direction and proceed to the next stage of batch descent.

\begin{quote}
\textbf{A TASK FOR YOU:} Perform 5 update steps of Stochastic Gradient
Descent with batchsize = \(1\) on our small data set. Recall that our
initial parameters were:
\[ \boldsymbol{\theta} = \begin{bmatrix} 1 \ \quad 0 \ \end{bmatrix}\]\\
\ldots{} and we used a learning rate of \(\boldsymbol{\eta} = 0.1\)
\end{quote}

(\(\eta\) is pronounced `eh-ta', sometimes we also use \(\alpha\),
``apha'' to denote learning rate, the two are equivalent)

Hand Calculations:

\begin{longtable}[]{@{}cccccc@{}}
\toprule
\begin{minipage}[b]{0.06\columnwidth}\centering
\(x_j '\)\strut
\end{minipage} & \begin{minipage}[b]{0.07\columnwidth}\centering
\(y_j\)\strut
\end{minipage} & \begin{minipage}[b]{0.17\columnwidth}\centering
\(\boldsymbol{\theta}\cdot\mathbf{x}'_j\)\strut
\end{minipage} & \begin{minipage}[b]{0.25\columnwidth}\centering
\(\frac{2}{B}\left[ \boldsymbol{\theta}^T\cdot\mathbf{x}'_j - y_j\right]\cdot\mathbf{x}'_j\)\strut
\end{minipage} & \begin{minipage}[b]{0.15\columnwidth}\centering
\(\eta \nabla_{\boldsymbol{\theta}} f\)\strut
\end{minipage} & \begin{minipage}[b]{0.13\columnwidth}\centering
\$\boldsymbol(\theta) - \eta \nabla\_\{\boldsymbol{\theta}\} f \$\strut
\end{minipage}\tabularnewline
\midrule
\endhead
\begin{minipage}[t]{0.06\columnwidth}\centering
input\strut
\end{minipage} & \begin{minipage}[t]{0.07\columnwidth}\centering
true y\strut
\end{minipage} & \begin{minipage}[t]{0.17\columnwidth}\centering
predicted y\strut
\end{minipage} & \begin{minipage}[t]{0.25\columnwidth}\centering
gradient for this `batch'\strut
\end{minipage} & \begin{minipage}[t]{0.15\columnwidth}\centering
update\strut
\end{minipage} & \begin{minipage}[t]{0.13\columnwidth}\centering
new parameters\strut
\end{minipage}\tabularnewline
\bottomrule
\end{longtable}

\textbar{} \$

\begin{bmatrix} 1 \\ 1 \\ \end{bmatrix}

\$ \textbar{} 2 \textbar{} \textbar{}\\
\textbar{} \$

\begin{bmatrix} 3 \\ 1 \\ \end{bmatrix}

\$ \textbar{} 4 \textbar{} \textbar{}\\
\textbar{} \$

\begin{bmatrix} 5 \\ 1 \\ \end{bmatrix}

\$ \textbar{} 5 \textbar{} \textbar{}\\
\textbar{} \$

\begin{bmatrix} 4 \\ 1 \\ \end{bmatrix}

\$ \textbar{} 3 \textbar{} \textbar{}\\
\textbar{} \$

\begin{bmatrix} 2 \\ 1 \\ \end{bmatrix}

\$ \textbar{} 3 \textbar{} \textbar{}

    \begin{quote}
\textbf{DISCUSSION QUESTIONS:}\\
* \emph{How does this result compare to our result from the hand
calculations in the last section? What implications does this have for
our quest to find the optimal paramters?} * \emph{How will parallelizing
Stoichastic Gradient Descent be similar/different to parallelizing
regular GD?}
\end{quote}

    \begin{Verbatim}[commandchars=\\\{\}]
{\color{incolor}In [{\color{incolor}51}]:} \PY{n}{Image}\PY{p}{(}\PY{n}{filename}\PY{o}{=}\PY{l+s+s2}{\PYZdq{}}\PY{l+s+s2}{demo6\PYZhy{}SGD.png}\PY{l+s+s2}{\PYZdq{}}\PY{p}{)}
\end{Verbatim}

\texttt{\color{outcolor}Out[{\color{outcolor}51}]:}
    
    \begin{center}
    \adjustimage{max size={0.9\linewidth}{0.9\paperheight}}{output_52_0.png}
    \end{center}
    { \hspace*{\fill} \\}
    

    \hypertarget{solution}{%
\subsubsection{\textless{}--- SOLUTION
---\textgreater{}}\label{solution}}

\textbf{INSTRUCTOR TALKING POINTS} * \emph{How does this result compare
to our result from the hand calculations in the last section? What
implications does this have for our quest to find the optimal
paramters?} \textgreater{} Although the first few individual updates
seem to go in odd directions by then end of one pass through the data
we've got a much better model than the equivalent model after 1 GD step
over the full dataset. This suggests that trainging via SGD will require
fewer passes over the data. * \emph{How will parallelizing Stochastic
Gradient Descent be similar/different to parallelizing regular GD?}
\textgreater{} With SGD, each data point produces a new parameter update
for use in the subsequent computation of the next data point. We can see
that SGD is by its nature a sequential algorithm, and paralellizing it
is a challenge in a map reduce framework.

    \begin{Verbatim}[commandchars=\\\{\}]
{\color{incolor}In [{\color{incolor}58}]:} \PY{k+kn}{import} \PY{n+nn}{helperFunc}\PY{o}{,} \PY{n+nn}{linRegFunc}
         
         \PY{c+c1}{\PYZsh{}\PYZsh{}\PYZsh{}\PYZsh{}\PYZsh{}\PYZsh{}\PYZsh{}\PYZsh{}\PYZsh{}\PYZsh{}\PYZsh{}\PYZsh{}\PYZsh{}\PYZsh{}\PYZsh{}\PYZsh{}\PYZsh{}\PYZsh{}\PYZsh{}\PYZsh{} Demo Parameters \PYZsh{}\PYZsh{}\PYZsh{}\PYZsh{}\PYZsh{}\PYZsh{}\PYZsh{}\PYZsh{}\PYZsh{}\PYZsh{}\PYZsh{}\PYZsh{}\PYZsh{}\PYZsh{}\PYZsh{}\PYZsh{}\PYZsh{}\PYZsh{}\PYZsh{}\PYZsh{} }
         \PY{c+c1}{\PYZsh{} TRY CHANGING THESE \PYZam{} SEE HOW IT AFFECTS OUR SEARCH}
         \PY{n}{N\PYZus{}STEPS} \PY{o}{=} \PY{l+m+mi}{5}
         \PY{n}{BATCHSIZE} \PY{o}{=} \PY{l+m+mi}{1}
         \PY{n}{LEARNING\PYZus{}RATE} \PY{o}{=} \PY{l+m+mf}{0.1}
         \PY{n}{ORIGINAL\PYZus{}MODEL} \PY{o}{=} \PY{p}{[}\PY{l+m+mi}{1}\PY{p}{,}\PY{l+m+mi}{0}\PY{p}{]}
         \PY{n}{SHOW\PYZus{}CONTOURS} \PY{o}{=} \PY{k+kc}{True}
         
         \PY{c+c1}{\PYZsh{}\PYZsh{}\PYZsh{}\PYZsh{}\PYZsh{}\PYZsh{}\PYZsh{}\PYZsh{}\PYZsh{}\PYZsh{}\PYZsh{}\PYZsh{}\PYZsh{}\PYZsh{}\PYZsh{}\PYZsh{}\PYZsh{}\PYZsh{}\PYZsh{} Stoichastic GD Demo \PYZsh{}\PYZsh{}\PYZsh{}\PYZsh{}\PYZsh{}\PYZsh{}\PYZsh{}\PYZsh{}\PYZsh{}\PYZsh{}\PYZsh{}\PYZsh{}\PYZsh{}\PYZsh{}\PYZsh{}\PYZsh{}\PYZsh{}\PYZsh{}\PYZsh{}\PYZsh{} }
         \PY{c+c1}{\PYZsh{}\PYZsh{}\PYZsh{} Load \PYZam{} Pre\PYZhy{}process Data}
         \PY{n}{points} \PY{o}{=} \PY{n}{np}\PY{o}{.}\PY{n}{genfromtxt}\PY{p}{(}\PY{l+s+s2}{\PYZdq{}}\PY{l+s+s2}{fivePoints.csv}\PY{l+s+s2}{\PYZdq{}}\PY{p}{,} \PY{n}{delimiter}\PY{o}{=}\PY{l+s+s1}{\PYZsq{}}\PY{l+s+s1}{,}\PY{l+s+s1}{\PYZsq{}}\PY{p}{)}
         \PY{n}{X} \PY{o}{=} \PY{n}{helperFunc}\PY{o}{.}\PY{n}{augment}\PY{p}{(}\PY{n}{points}\PY{p}{)}\PY{p}{[}\PY{p}{:}\PY{p}{,}\PY{p}{:}\PY{l+m+mi}{2}\PY{p}{]}
         \PY{n}{y} \PY{o}{=} \PY{n}{points}\PY{p}{[}\PY{p}{:}\PY{p}{,}\PY{l+m+mi}{1}\PY{p}{]}
         
         \PY{c+c1}{\PYZsh{}\PYZsh{}\PYZsh{} Perform SGD Updates \PYZam{} save intermediate model performance}
         \PY{n}{models}\PY{p}{,} \PY{n}{loss} \PY{o}{=} \PY{n}{linRegFunc}\PY{o}{.}\PY{n}{SGDUpdate}\PY{p}{(}\PY{n}{X}\PY{p}{,} \PY{n}{y}\PY{p}{,} \PY{n}{N\PYZus{}STEPS}\PY{p}{,}
                                             \PY{n}{BATCHSIZE}\PY{p}{,}
                                             \PY{n}{ORIGINAL\PYZus{}MODEL}\PY{p}{,} 
                                             \PY{n}{LEARNING\PYZus{}RATE}\PY{p}{,} 
                                             \PY{n}{verbose} \PY{o}{=} \PY{k+kc}{False}\PY{p}{)}
         
         \PY{c+c1}{\PYZsh{}\PYZsh{}\PYZsh{} Display Results}
         \PY{n+nb}{print}\PY{p}{(}\PY{n}{f}\PY{l+s+s2}{\PYZdq{}}\PY{l+s+se}{\PYZbs{}n}\PY{l+s+s2}{Searched }\PY{l+s+s2}{\PYZob{}}\PY{l+s+s2}{len(models)\PYZcb{} models...}\PY{l+s+s2}{\PYZdq{}} \PY{o}{\PYZpc{}}\PY{p}{(}\PY{p}{)}\PY{p}{)}
         \PY{n+nb}{print}\PY{p}{(}\PY{n}{f}\PY{l+s+s2}{\PYZdq{}}\PY{l+s+s2}{Best model: }\PY{l+s+si}{\PYZob{}models[np.argmin(loss)]\PYZcb{}}\PY{l+s+s2}{, Loss: }\PY{l+s+si}{\PYZob{}loss[np.argmin(loss)]\PYZcb{}}\PY{l+s+s2}{\PYZdq{}}\PY{p}{)}
         \PY{n}{linRegFunc}\PY{o}{.}\PY{n}{plotGDProgress}\PY{p}{(}\PY{n}{points}\PY{p}{,} \PY{n}{models}\PY{p}{,} \PY{n}{loss}\PY{p}{,}
                                   \PY{n}{show\PYZus{}contours} \PY{o}{=} \PY{n}{SHOW\PYZus{}CONTOURS}\PY{p}{)}
\end{Verbatim}


    \begin{Verbatim}[commandchars=\\\{\}]

Searched 6 models{\ldots}
Best model: [1. 0.], Loss: 0.8

    \end{Verbatim}

    \begin{center}
    \adjustimage{max size={0.9\linewidth}{0.9\paperheight}}{output_54_1.png}
    \end{center}
    { \hspace*{\fill} \\}
    
    \begin{quote}
\textbf{DISCUSSION QUESTIONS:} * \emph{At first glance does this seem to
work as well as regular gradient descent? Why might our initial
impression be deceiving?} * \emph{Does adjusting the batchsize and/or
learning rate fix the problem that we're seeing?} * \emph{What do you
notice about the direction of the first 3 updates? From the perspective
of the first three points, what should our line look like?} * \emph{How
does the scale of our data can impact the direction of our updates \&
time to convergence?}
\end{quote}

    \hypertarget{solution}{%
\subsubsection{\textless{}--- SOLUTION
---\textgreater{}}\label{solution}}

\textbf{INSTRUCTOR TALKING POINTS} * \emph{At first glance does this
seem to work as well as regular gradient descent? Why might our initial
impression be deceiving?} \textgreater{} SGD will oscillate a lot so at
first glance it looks very unstable - as if it is not traveling in the
direction of the minimum. * \emph{Does adjusting the batchsize and/or
learning rate fix the problem that we're seeing?} \textgreater{} For
such a small dataset adjusting the batchsize and/or learning rate won't
be very helpful. * \emph{What do you notice about the direction of the
first 3 updates? From the perspective of the first three points, what
should our line look like?} \textgreater{} The first thre points form a
line in an upward right direction. Without seeing the other points, one
might think this line would continue up towards the solution. *
\emph{How does the scale of our data impact the direction of our updates
\& time to convergence?} \textgreater{} In Stochastic GD, the scale of
the data does not impact direction of the updates as the updates are
made at each data point. Compared to Batch GD a large data set will need
fewer passes (epochs) over the entire dataset to converge.

    \textbf{For more info, here are a few of rabbit holes:} \textgreater{}
https://arxiv.org/pdf/1707.00424.pdf\\
\textgreater{}
https://papers.nips.cc/paper/4390-hogwild-a-lock-free-approach-to-parallelizing-stochastic-gradient-descent.pdf
\textgreater{}
http://papers.nips.cc/paper/4006-parallelized-stochastic-gradient-descent.pdf\\
\textgreater{}
https://www.microsoft.com/en-us/research/wp-content/uploads/2016/02/distr\_mini\_batch.pdf

    \hypertarget{thats-it-for-today}{%
\subsubsection{That's it for today!}\label{thats-it-for-today}}

\hypertarget{next-week-we-will-discuss}{%
\paragraph{Next week we will
discuss\ldots{}}\label{next-week-we-will-discuss}}

\begin{itemize}
\tightlist
\item
  \textbf{L1 and L2 Regularization}
\item
  \textbf{Common GD variants}
\item
  \textbf{What to do if you can't compute a gradient for your loss
  function.}
\item
  \textbf{Logistic Regression \& classification}
\end{itemize}


    % Add a bibliography block to the postdoc
    
    
    
    \end{document}
